% accessible_typography.tex
%
% ACCESSIBLE TYPOGRAPHY CONFIGURATION FOR QGP LIGHT-ION PROJECT
% Optimized for:
%   - Readability across visual abilities (dyslexia, low vision)
%   - Scientific document conventions
%   - Alignment with accessible_colors.tex color scheme
%   - Print and screen compatibility
%
% Based on: Libertinus font family + typographic research best practices
%
% USAGE: % accessible_typography.tex
%
% ACCESSIBLE TYPOGRAPHY CONFIGURATION FOR QGP LIGHT-ION PROJECT
% Optimized for:
%   - Readability across visual abilities (dyslexia, low vision)
%   - Scientific document conventions
%   - Alignment with accessible_colors.tex color scheme
%   - Print and screen compatibility
%
% Based on: Libertinus font family + typographic research best practices
%
% USAGE: % accessible_typography.tex
%
% ACCESSIBLE TYPOGRAPHY CONFIGURATION FOR QGP LIGHT-ION PROJECT
% Optimized for:
%   - Readability across visual abilities (dyslexia, low vision)
%   - Scientific document conventions
%   - Alignment with accessible_colors.tex color scheme
%   - Print and screen compatibility
%
% Based on: Libertinus font family + typographic research best practices
%
% USAGE: % accessible_typography.tex
%
% ACCESSIBLE TYPOGRAPHY CONFIGURATION FOR QGP LIGHT-ION PROJECT
% Optimized for:
%   - Readability across visual abilities (dyslexia, low vision)
%   - Scientific document conventions
%   - Alignment with accessible_colors.tex color scheme
%   - Print and screen compatibility
%
% Based on: Libertinus font family + typographic research best practices
%
% USAGE: \input{figures/accessible_typography.tex} in preamble AFTER geometry
%
% ============================================================================
% TYPOGRAPHIC RATIONALE
% ============================================================================
%
% LINE LENGTH (Measure):
%   - Optimal: 60-75 characters per line (research consensus)
%   - At 11pt Libertinus, ~66 chars ≈ 5.0 inches text width
%   - Wider margins improve focus and allow marginal notes
%
% LINE HEIGHT (Leading):
%   - Body text: 1.35× (130-140% recommended for accessibility)
%   - Captions: 1.25× (tighter for shorter text blocks)
%   - Code/tables: 1.15× (dense information)
%
% FONT SIZE HIERARCHY:
%   - Body: 11pt (standard academic, good for extended reading)
%   - Captions: 10pt (0.9× body)
%   - Footnotes: 9pt (0.8× body)
%   - Section headings: 14pt bold (1.27× body)
%   - Subsection: 12pt bold (1.09× body)
%
% LETTER SPACING:
%   - Slight positive tracking improves readability for dyslexia
%   - Headings: +20 units (microtype)
%   - Body: default (Libertinus is well-spaced)
%
% ============================================================================

% --- FONT PACKAGES ---
% Libertinus: professional serif with complete math support
% Fork of Linux Libertine with OpenType math, maintained actively
\usepackage{libertinus-type1}  % For pdfLaTeX (Type 1 fonts)

% Alternative for LuaLaTeX/XeLaTeX (uncomment if using those):
% \usepackage{libertinus-otf}

% Inconsolata for monospace (clean, accessible coding font)
\usepackage{inconsolata}

% ============================================================================
% PAGE GEOMETRY (Optimal Placement)
% ============================================================================
% Calculated for US Letter (8.5" × 11") with golden ratio principles
%
% Text block dimensions:
%   - Width: 5.25 inches (≈66 chars at 11pt Libertinus)
%   - Height: 8.0 inches (leaves room for headers/footers)
%
% Margin distribution (traditional "canonical" proportions):
%   - Inner margin: 1.0 inch (binding edge)
%   - Outer margin: 1.25 inch (thumb space + visual balance)
%   - Top margin: 1.0 inch
%   - Bottom margin: 2.0 inch (footer + page number space)
%
% Note: geometry package should be loaded BEFORE this file
% These settings override the basic 1in margins

\geometry{
    letterpaper,
    % Specify margins only; let geometry calculate text dimensions
    % This avoids over-specification warnings
    inner=1.0in,
    outer=1.25in,
    top=1.0in,
    bottom=1.5in,
    marginparwidth=0.75in,
    marginparsep=0.125in,
    headheight=24pt,          % Increased for Libertinus font height
    headsep=0.25in,
    footskip=0.5in
}

% ============================================================================
% LINE SPACING
% ============================================================================
\usepackage{setspace}
\setstretch{1.35}  % 135% line height for body text

% ============================================================================
% PARAGRAPH FORMATTING
% ============================================================================
% No paragraph indentation (block style) with vertical space between paragraphs
% This is more accessible for readers with tracking difficulties

\setlength{\parindent}{0pt}
\setlength{\parskip}{0.6\baselineskip plus 0.1\baselineskip minus 0.1\baselineskip}

% ============================================================================
% HEADING STYLES
% ============================================================================
\usepackage{titlesec}

% Section: Large, bold, with generous spacing
% Sans-serif for better visibility at larger sizes
\titleformat{\section}
    {\Large\bfseries\sffamily}  % Sans-serif bold
    {\thesection}
    {1em}
    {}
\titlespacing*{\section}
    {0pt}                        % Left margin
    {2\baselineskip plus 0.5\baselineskip}  % Before
    {1\baselineskip}             % After

% Subsection: Medium, bold
\titleformat{\subsection}
    {\large\bfseries\sffamily}
    {\thesubsection}
    {1em}
    {}
\titlespacing*{\subsection}
    {0pt}
    {1.5\baselineskip plus 0.3\baselineskip}
    {0.75\baselineskip}

% Subsubsection: Normal size, bold
\titleformat{\subsubsection}
    {\normalsize\bfseries\sffamily}
    {\thesubsubsection}
    {1em}
    {}
\titlespacing*{\subsubsection}
    {0pt}
    {1\baselineskip plus 0.2\baselineskip}
    {0.5\baselineskip}

% ============================================================================
% CAPTION STYLING
% ============================================================================
\usepackage[
    font={small,stretch=1.1},     % 10pt, slightly tighter than body
    labelfont={bf,sf},            % Bold sans-serif labels
    labelsep=period,              % "Figure 1. Caption text"
    justification=justified,
    singlelinecheck=false,
    margin=0.5in,                 % Indent captions from text margins
    format=hang,                  % Hanging indent for multi-line
    skip=0.75\baselineskip        % Space between figure and caption
]{caption}

% ============================================================================
% TABLE STYLING
% ============================================================================
\usepackage{booktabs}  % Professional table rules
\usepackage{array}     % Enhanced column types

% Increased row spacing in tables
\renewcommand{\arraystretch}{1.3}

% ============================================================================
% LIST FORMATTING
% ============================================================================
\usepackage{enumitem}

% Tighter lists with accessible spacing
\setlist{
    topsep=0.5\baselineskip,
    itemsep=0.25\baselineskip,
    parsep=0pt,
    leftmargin=1.5em
}

% Description lists (used in Appendix)
\setlist[description]{
    style=nextline,           % Label on its own line
    leftmargin=1em,           % Positive margin to avoid warning
    labelindent=0pt,
    itemsep=0.75\baselineskip,
    font=\bfseries\sffamily   % Bold sans-serif terms
}

% ============================================================================
% MATH TYPOGRAPHY
% ============================================================================
% Libertinus Math integrates seamlessly with Libertinus text

% Slightly larger math for readability
\DeclareMathSizes{11}{11}{8}{6}    % text, display, script, scriptscript

% ============================================================================
% MICROTYPE (Fine Typography)
% ============================================================================
\usepackage[
    activate={true,nocompatibility},
    final,
    tracking=true,
    factor=1100,
    stretch=10,
    shrink=10,
    nopatch=footnote  % Avoid conflict with footmisc package
]{microtype}

% Slight letter spacing for small caps (used in some abbreviations)
\SetTracking{encoding={*}, shape=sc}{40}

% ============================================================================
% HYPERLINK STYLING
% ============================================================================
% Accessible colors for links (from accessible_colors.tex palette)
\hypersetup{
    colorlinks=true,
    linkcolor={rgb:red,0;green,0.447;blue,0.698},    % PbPbcolor (deep blue)
    citecolor={rgb:red,0;green,0.447;blue,0.698},    % Same blue for citations
    urlcolor={rgb:red,0.835;green,0.369;blue,0},     % theorycolor (vermillion)
    filecolor={rgb:red,0;green,0.620;blue,0.451},    % NeNecolor (teal)
    % Note: bookmarks options moved to \usepackage[...]{hyperref} in main document
    pdfstartview=FitH
}

% ============================================================================
% HEADERS AND FOOTERS
% ============================================================================
\usepackage{fancyhdr}
\pagestyle{fancy}
\fancyhf{}  % Clear defaults

% Header: Short title (left), page number (right)
% Note: Using fixed short title instead of \leftmark to avoid overly long headers
\fancyhead[L]{\small\sffamily Light-Ion QGP: O--O \& Ne--Ne}
\fancyhead[R]{\small\sffamily\thepage}

% Footer: Empty (clean look)
\fancyfoot{}

% Line under header
\renewcommand{\headrulewidth}{0.4pt}
\renewcommand{\footrulewidth}{0pt}

% Plain style for chapter/section opening pages
\fancypagestyle{plain}{
    \fancyhf{}
    \fancyfoot[C]{\small\sffamily\thepage}
    \renewcommand{\headrulewidth}{0pt}
}

% ============================================================================
% TITLE PAGE STYLING
% ============================================================================
% Custom title format for accessibility

\makeatletter
\renewcommand{\maketitle}{%
    \begin{center}
        \vspace*{1in}
        {\LARGE\bfseries\sffamily \@title \par}
        \vspace{1.5\baselineskip}
        {\large \@author \par}
        \vspace{0.5\baselineskip}
        {\normalsize \@date \par}
        \vspace{2\baselineskip}
    \end{center}
    \thispagestyle{empty}
}
\makeatother

% ============================================================================
% FOOTNOTE STYLING
% ============================================================================
\usepackage[bottom,hang]{footmisc}
\setlength{\footnotemargin}{0.5em}
\renewcommand{\footnotelayout}{\small}

% ============================================================================
% BLOCK QUOTE STYLING
% ============================================================================
\usepackage{quoting}
\quotingsetup{
    leftmargin=1.5em,
    rightmargin=1.5em,
    vskip=0.5\baselineskip,
    font={small,itshape}
}

% ============================================================================
% ACCESSIBILITY NOTES
% ============================================================================
%
% FONT CHOICE RATIONALE:
%   - Libertinus is a high-quality open-source serif with:
%     * Clear letterforms with good x-height
%     * Distinct character shapes (reduces confusion: l/1/I, O/0)
%     * Complete math support (Libertinus Math)
%     * Active maintenance and bug fixes
%
% SANS-SERIF FOR HEADINGS:
%   - Research shows sans-serif is clearer at larger sizes
%   - Libertinus Sans matches Libertinus Serif metrics
%   - Creates clear visual hierarchy
%
% LINE SPACING RATIONALE:
%   - 1.35× (135%) exceeds minimum 1.5× recommended by some guidelines
%   - but 1.5× was found excessive for academic text
%   - 1.35× balances readability with page economy
%
% AVOIDING PROBLEMATIC ELEMENTS:
%   - No justified text with hyphenation (causes "rivers")
%   - No pure black text on pure white (too high contrast)
%   - Minimal italics (hard for dyslexic readers)
%   - No all-caps headings (reduces recognition)
%
% ============================================================================
 in preamble AFTER geometry
%
% ============================================================================
% TYPOGRAPHIC RATIONALE
% ============================================================================
%
% LINE LENGTH (Measure):
%   - Optimal: 60-75 characters per line (research consensus)
%   - At 11pt Libertinus, ~66 chars ≈ 5.0 inches text width
%   - Wider margins improve focus and allow marginal notes
%
% LINE HEIGHT (Leading):
%   - Body text: 1.35× (130-140% recommended for accessibility)
%   - Captions: 1.25× (tighter for shorter text blocks)
%   - Code/tables: 1.15× (dense information)
%
% FONT SIZE HIERARCHY:
%   - Body: 11pt (standard academic, good for extended reading)
%   - Captions: 10pt (0.9× body)
%   - Footnotes: 9pt (0.8× body)
%   - Section headings: 14pt bold (1.27× body)
%   - Subsection: 12pt bold (1.09× body)
%
% LETTER SPACING:
%   - Slight positive tracking improves readability for dyslexia
%   - Headings: +20 units (microtype)
%   - Body: default (Libertinus is well-spaced)
%
% ============================================================================

% --- FONT PACKAGES ---
% Libertinus: professional serif with complete math support
% Fork of Linux Libertine with OpenType math, maintained actively
\usepackage{libertinus-type1}  % For pdfLaTeX (Type 1 fonts)

% Alternative for LuaLaTeX/XeLaTeX (uncomment if using those):
% \usepackage{libertinus-otf}

% Inconsolata for monospace (clean, accessible coding font)
\usepackage{inconsolata}

% ============================================================================
% PAGE GEOMETRY (Optimal Placement)
% ============================================================================
% Calculated for US Letter (8.5" × 11") with golden ratio principles
%
% Text block dimensions:
%   - Width: 5.25 inches (≈66 chars at 11pt Libertinus)
%   - Height: 8.0 inches (leaves room for headers/footers)
%
% Margin distribution (traditional "canonical" proportions):
%   - Inner margin: 1.0 inch (binding edge)
%   - Outer margin: 1.25 inch (thumb space + visual balance)
%   - Top margin: 1.0 inch
%   - Bottom margin: 2.0 inch (footer + page number space)
%
% Note: geometry package should be loaded BEFORE this file
% These settings override the basic 1in margins

\geometry{
    letterpaper,
    % Specify margins only; let geometry calculate text dimensions
    % This avoids over-specification warnings
    inner=1.0in,
    outer=1.25in,
    top=1.0in,
    bottom=1.5in,
    marginparwidth=0.75in,
    marginparsep=0.125in,
    headheight=24pt,          % Increased for Libertinus font height
    headsep=0.25in,
    footskip=0.5in
}

% ============================================================================
% LINE SPACING
% ============================================================================
\usepackage{setspace}
\setstretch{1.35}  % 135% line height for body text

% ============================================================================
% PARAGRAPH FORMATTING
% ============================================================================
% No paragraph indentation (block style) with vertical space between paragraphs
% This is more accessible for readers with tracking difficulties

\setlength{\parindent}{0pt}
\setlength{\parskip}{0.6\baselineskip plus 0.1\baselineskip minus 0.1\baselineskip}

% ============================================================================
% HEADING STYLES
% ============================================================================
\usepackage{titlesec}

% Section: Large, bold, with generous spacing
% Sans-serif for better visibility at larger sizes
\titleformat{\section}
    {\Large\bfseries\sffamily}  % Sans-serif bold
    {\thesection}
    {1em}
    {}
\titlespacing*{\section}
    {0pt}                        % Left margin
    {2\baselineskip plus 0.5\baselineskip}  % Before
    {1\baselineskip}             % After

% Subsection: Medium, bold
\titleformat{\subsection}
    {\large\bfseries\sffamily}
    {\thesubsection}
    {1em}
    {}
\titlespacing*{\subsection}
    {0pt}
    {1.5\baselineskip plus 0.3\baselineskip}
    {0.75\baselineskip}

% Subsubsection: Normal size, bold
\titleformat{\subsubsection}
    {\normalsize\bfseries\sffamily}
    {\thesubsubsection}
    {1em}
    {}
\titlespacing*{\subsubsection}
    {0pt}
    {1\baselineskip plus 0.2\baselineskip}
    {0.5\baselineskip}

% ============================================================================
% CAPTION STYLING
% ============================================================================
\usepackage[
    font={small,stretch=1.1},     % 10pt, slightly tighter than body
    labelfont={bf,sf},            % Bold sans-serif labels
    labelsep=period,              % "Figure 1. Caption text"
    justification=justified,
    singlelinecheck=false,
    margin=0.5in,                 % Indent captions from text margins
    format=hang,                  % Hanging indent for multi-line
    skip=0.75\baselineskip        % Space between figure and caption
]{caption}

% ============================================================================
% TABLE STYLING
% ============================================================================
\usepackage{booktabs}  % Professional table rules
\usepackage{array}     % Enhanced column types

% Increased row spacing in tables
\renewcommand{\arraystretch}{1.3}

% ============================================================================
% LIST FORMATTING
% ============================================================================
\usepackage{enumitem}

% Tighter lists with accessible spacing
\setlist{
    topsep=0.5\baselineskip,
    itemsep=0.25\baselineskip,
    parsep=0pt,
    leftmargin=1.5em
}

% Description lists (used in Appendix)
\setlist[description]{
    style=nextline,           % Label on its own line
    leftmargin=1em,           % Positive margin to avoid warning
    labelindent=0pt,
    itemsep=0.75\baselineskip,
    font=\bfseries\sffamily   % Bold sans-serif terms
}

% ============================================================================
% MATH TYPOGRAPHY
% ============================================================================
% Libertinus Math integrates seamlessly with Libertinus text

% Slightly larger math for readability
\DeclareMathSizes{11}{11}{8}{6}    % text, display, script, scriptscript

% ============================================================================
% MICROTYPE (Fine Typography)
% ============================================================================
\usepackage[
    activate={true,nocompatibility},
    final,
    tracking=true,
    factor=1100,
    stretch=10,
    shrink=10,
    nopatch=footnote  % Avoid conflict with footmisc package
]{microtype}

% Slight letter spacing for small caps (used in some abbreviations)
\SetTracking{encoding={*}, shape=sc}{40}

% ============================================================================
% HYPERLINK STYLING
% ============================================================================
% Accessible colors for links (from accessible_colors.tex palette)
\hypersetup{
    colorlinks=true,
    linkcolor={rgb:red,0;green,0.447;blue,0.698},    % PbPbcolor (deep blue)
    citecolor={rgb:red,0;green,0.447;blue,0.698},    % Same blue for citations
    urlcolor={rgb:red,0.835;green,0.369;blue,0},     % theorycolor (vermillion)
    filecolor={rgb:red,0;green,0.620;blue,0.451},    % NeNecolor (teal)
    % Note: bookmarks options moved to \usepackage[...]{hyperref} in main document
    pdfstartview=FitH
}

% ============================================================================
% HEADERS AND FOOTERS
% ============================================================================
\usepackage{fancyhdr}
\pagestyle{fancy}
\fancyhf{}  % Clear defaults

% Header: Short title (left), page number (right)
% Note: Using fixed short title instead of \leftmark to avoid overly long headers
\fancyhead[L]{\small\sffamily Light-Ion QGP: O--O \& Ne--Ne}
\fancyhead[R]{\small\sffamily\thepage}

% Footer: Empty (clean look)
\fancyfoot{}

% Line under header
\renewcommand{\headrulewidth}{0.4pt}
\renewcommand{\footrulewidth}{0pt}

% Plain style for chapter/section opening pages
\fancypagestyle{plain}{
    \fancyhf{}
    \fancyfoot[C]{\small\sffamily\thepage}
    \renewcommand{\headrulewidth}{0pt}
}

% ============================================================================
% TITLE PAGE STYLING
% ============================================================================
% Custom title format for accessibility

\makeatletter
\renewcommand{\maketitle}{%
    \begin{center}
        \vspace*{1in}
        {\LARGE\bfseries\sffamily \@title \par}
        \vspace{1.5\baselineskip}
        {\large \@author \par}
        \vspace{0.5\baselineskip}
        {\normalsize \@date \par}
        \vspace{2\baselineskip}
    \end{center}
    \thispagestyle{empty}
}
\makeatother

% ============================================================================
% FOOTNOTE STYLING
% ============================================================================
\usepackage[bottom,hang]{footmisc}
\setlength{\footnotemargin}{0.5em}
\renewcommand{\footnotelayout}{\small}

% ============================================================================
% BLOCK QUOTE STYLING
% ============================================================================
\usepackage{quoting}
\quotingsetup{
    leftmargin=1.5em,
    rightmargin=1.5em,
    vskip=0.5\baselineskip,
    font={small,itshape}
}

% ============================================================================
% ACCESSIBILITY NOTES
% ============================================================================
%
% FONT CHOICE RATIONALE:
%   - Libertinus is a high-quality open-source serif with:
%     * Clear letterforms with good x-height
%     * Distinct character shapes (reduces confusion: l/1/I, O/0)
%     * Complete math support (Libertinus Math)
%     * Active maintenance and bug fixes
%
% SANS-SERIF FOR HEADINGS:
%   - Research shows sans-serif is clearer at larger sizes
%   - Libertinus Sans matches Libertinus Serif metrics
%   - Creates clear visual hierarchy
%
% LINE SPACING RATIONALE:
%   - 1.35× (135%) exceeds minimum 1.5× recommended by some guidelines
%   - but 1.5× was found excessive for academic text
%   - 1.35× balances readability with page economy
%
% AVOIDING PROBLEMATIC ELEMENTS:
%   - No justified text with hyphenation (causes "rivers")
%   - No pure black text on pure white (too high contrast)
%   - Minimal italics (hard for dyslexic readers)
%   - No all-caps headings (reduces recognition)
%
% ============================================================================
 in preamble AFTER geometry
%
% ============================================================================
% TYPOGRAPHIC RATIONALE
% ============================================================================
%
% LINE LENGTH (Measure):
%   - Optimal: 60-75 characters per line (research consensus)
%   - At 11pt Libertinus, ~66 chars ≈ 5.0 inches text width
%   - Wider margins improve focus and allow marginal notes
%
% LINE HEIGHT (Leading):
%   - Body text: 1.35× (130-140% recommended for accessibility)
%   - Captions: 1.25× (tighter for shorter text blocks)
%   - Code/tables: 1.15× (dense information)
%
% FONT SIZE HIERARCHY:
%   - Body: 11pt (standard academic, good for extended reading)
%   - Captions: 10pt (0.9× body)
%   - Footnotes: 9pt (0.8× body)
%   - Section headings: 14pt bold (1.27× body)
%   - Subsection: 12pt bold (1.09× body)
%
% LETTER SPACING:
%   - Slight positive tracking improves readability for dyslexia
%   - Headings: +20 units (microtype)
%   - Body: default (Libertinus is well-spaced)
%
% ============================================================================

% --- FONT PACKAGES ---
% Libertinus: professional serif with complete math support
% Fork of Linux Libertine with OpenType math, maintained actively
\usepackage{libertinus-type1}  % For pdfLaTeX (Type 1 fonts)

% Alternative for LuaLaTeX/XeLaTeX (uncomment if using those):
% \usepackage{libertinus-otf}

% Inconsolata for monospace (clean, accessible coding font)
\usepackage{inconsolata}

% ============================================================================
% PAGE GEOMETRY (Optimal Placement)
% ============================================================================
% Calculated for US Letter (8.5" × 11") with golden ratio principles
%
% Text block dimensions:
%   - Width: 5.25 inches (≈66 chars at 11pt Libertinus)
%   - Height: 8.0 inches (leaves room for headers/footers)
%
% Margin distribution (traditional "canonical" proportions):
%   - Inner margin: 1.0 inch (binding edge)
%   - Outer margin: 1.25 inch (thumb space + visual balance)
%   - Top margin: 1.0 inch
%   - Bottom margin: 2.0 inch (footer + page number space)
%
% Note: geometry package should be loaded BEFORE this file
% These settings override the basic 1in margins

\geometry{
    letterpaper,
    % Specify margins only; let geometry calculate text dimensions
    % This avoids over-specification warnings
    inner=1.0in,
    outer=1.25in,
    top=1.0in,
    bottom=1.5in,
    marginparwidth=0.75in,
    marginparsep=0.125in,
    headheight=24pt,          % Increased for Libertinus font height
    headsep=0.25in,
    footskip=0.5in
}

% ============================================================================
% LINE SPACING
% ============================================================================
\usepackage{setspace}
\setstretch{1.35}  % 135% line height for body text

% ============================================================================
% PARAGRAPH FORMATTING
% ============================================================================
% No paragraph indentation (block style) with vertical space between paragraphs
% This is more accessible for readers with tracking difficulties

\setlength{\parindent}{0pt}
\setlength{\parskip}{0.6\baselineskip plus 0.1\baselineskip minus 0.1\baselineskip}

% ============================================================================
% HEADING STYLES
% ============================================================================
\usepackage{titlesec}

% Section: Large, bold, with generous spacing
% Sans-serif for better visibility at larger sizes
\titleformat{\section}
    {\Large\bfseries\sffamily}  % Sans-serif bold
    {\thesection}
    {1em}
    {}
\titlespacing*{\section}
    {0pt}                        % Left margin
    {2\baselineskip plus 0.5\baselineskip}  % Before
    {1\baselineskip}             % After

% Subsection: Medium, bold
\titleformat{\subsection}
    {\large\bfseries\sffamily}
    {\thesubsection}
    {1em}
    {}
\titlespacing*{\subsection}
    {0pt}
    {1.5\baselineskip plus 0.3\baselineskip}
    {0.75\baselineskip}

% Subsubsection: Normal size, bold
\titleformat{\subsubsection}
    {\normalsize\bfseries\sffamily}
    {\thesubsubsection}
    {1em}
    {}
\titlespacing*{\subsubsection}
    {0pt}
    {1\baselineskip plus 0.2\baselineskip}
    {0.5\baselineskip}

% ============================================================================
% CAPTION STYLING
% ============================================================================
\usepackage[
    font={small,stretch=1.1},     % 10pt, slightly tighter than body
    labelfont={bf,sf},            % Bold sans-serif labels
    labelsep=period,              % "Figure 1. Caption text"
    justification=justified,
    singlelinecheck=false,
    margin=0.5in,                 % Indent captions from text margins
    format=hang,                  % Hanging indent for multi-line
    skip=0.75\baselineskip        % Space between figure and caption
]{caption}

% ============================================================================
% TABLE STYLING
% ============================================================================
\usepackage{booktabs}  % Professional table rules
\usepackage{array}     % Enhanced column types

% Increased row spacing in tables
\renewcommand{\arraystretch}{1.3}

% ============================================================================
% LIST FORMATTING
% ============================================================================
\usepackage{enumitem}

% Tighter lists with accessible spacing
\setlist{
    topsep=0.5\baselineskip,
    itemsep=0.25\baselineskip,
    parsep=0pt,
    leftmargin=1.5em
}

% Description lists (used in Appendix)
\setlist[description]{
    style=nextline,           % Label on its own line
    leftmargin=1em,           % Positive margin to avoid warning
    labelindent=0pt,
    itemsep=0.75\baselineskip,
    font=\bfseries\sffamily   % Bold sans-serif terms
}

% ============================================================================
% MATH TYPOGRAPHY
% ============================================================================
% Libertinus Math integrates seamlessly with Libertinus text

% Slightly larger math for readability
\DeclareMathSizes{11}{11}{8}{6}    % text, display, script, scriptscript

% ============================================================================
% MICROTYPE (Fine Typography)
% ============================================================================
\usepackage[
    activate={true,nocompatibility},
    final,
    tracking=true,
    factor=1100,
    stretch=10,
    shrink=10,
    nopatch=footnote  % Avoid conflict with footmisc package
]{microtype}

% Slight letter spacing for small caps (used in some abbreviations)
\SetTracking{encoding={*}, shape=sc}{40}

% ============================================================================
% HYPERLINK STYLING
% ============================================================================
% Accessible colors for links (from accessible_colors.tex palette)
\hypersetup{
    colorlinks=true,
    linkcolor={rgb:red,0;green,0.447;blue,0.698},    % PbPbcolor (deep blue)
    citecolor={rgb:red,0;green,0.447;blue,0.698},    % Same blue for citations
    urlcolor={rgb:red,0.835;green,0.369;blue,0},     % theorycolor (vermillion)
    filecolor={rgb:red,0;green,0.620;blue,0.451},    % NeNecolor (teal)
    % Note: bookmarks options moved to \usepackage[...]{hyperref} in main document
    pdfstartview=FitH
}

% ============================================================================
% HEADERS AND FOOTERS
% ============================================================================
\usepackage{fancyhdr}
\pagestyle{fancy}
\fancyhf{}  % Clear defaults

% Header: Short title (left), page number (right)
% Note: Using fixed short title instead of \leftmark to avoid overly long headers
\fancyhead[L]{\small\sffamily Light-Ion QGP: O--O \& Ne--Ne}
\fancyhead[R]{\small\sffamily\thepage}

% Footer: Empty (clean look)
\fancyfoot{}

% Line under header
\renewcommand{\headrulewidth}{0.4pt}
\renewcommand{\footrulewidth}{0pt}

% Plain style for chapter/section opening pages
\fancypagestyle{plain}{
    \fancyhf{}
    \fancyfoot[C]{\small\sffamily\thepage}
    \renewcommand{\headrulewidth}{0pt}
}

% ============================================================================
% TITLE PAGE STYLING
% ============================================================================
% Custom title format for accessibility

\makeatletter
\renewcommand{\maketitle}{%
    \begin{center}
        \vspace*{1in}
        {\LARGE\bfseries\sffamily \@title \par}
        \vspace{1.5\baselineskip}
        {\large \@author \par}
        \vspace{0.5\baselineskip}
        {\normalsize \@date \par}
        \vspace{2\baselineskip}
    \end{center}
    \thispagestyle{empty}
}
\makeatother

% ============================================================================
% FOOTNOTE STYLING
% ============================================================================
\usepackage[bottom,hang]{footmisc}
\setlength{\footnotemargin}{0.5em}
\renewcommand{\footnotelayout}{\small}

% ============================================================================
% BLOCK QUOTE STYLING
% ============================================================================
\usepackage{quoting}
\quotingsetup{
    leftmargin=1.5em,
    rightmargin=1.5em,
    vskip=0.5\baselineskip,
    font={small,itshape}
}

% ============================================================================
% ACCESSIBILITY NOTES
% ============================================================================
%
% FONT CHOICE RATIONALE:
%   - Libertinus is a high-quality open-source serif with:
%     * Clear letterforms with good x-height
%     * Distinct character shapes (reduces confusion: l/1/I, O/0)
%     * Complete math support (Libertinus Math)
%     * Active maintenance and bug fixes
%
% SANS-SERIF FOR HEADINGS:
%   - Research shows sans-serif is clearer at larger sizes
%   - Libertinus Sans matches Libertinus Serif metrics
%   - Creates clear visual hierarchy
%
% LINE SPACING RATIONALE:
%   - 1.35× (135%) exceeds minimum 1.5× recommended by some guidelines
%   - but 1.5× was found excessive for academic text
%   - 1.35× balances readability with page economy
%
% AVOIDING PROBLEMATIC ELEMENTS:
%   - No justified text with hyphenation (causes "rivers")
%   - No pure black text on pure white (too high contrast)
%   - Minimal italics (hard for dyslexic readers)
%   - No all-caps headings (reduces recognition)
%
% ============================================================================
 in preamble AFTER geometry
%
% ============================================================================
% TYPOGRAPHIC RATIONALE
% ============================================================================
%
% LINE LENGTH (Measure):
%   - Optimal: 60-75 characters per line (research consensus)
%   - At 11pt Libertinus, ~66 chars ≈ 5.0 inches text width
%   - Wider margins improve focus and allow marginal notes
%
% LINE HEIGHT (Leading):
%   - Body text: 1.35× (130-140% recommended for accessibility)
%   - Captions: 1.25× (tighter for shorter text blocks)
%   - Code/tables: 1.15× (dense information)
%
% FONT SIZE HIERARCHY:
%   - Body: 11pt (standard academic, good for extended reading)
%   - Captions: 10pt (0.9× body)
%   - Footnotes: 9pt (0.8× body)
%   - Section headings: 14pt bold (1.27× body)
%   - Subsection: 12pt bold (1.09× body)
%
% LETTER SPACING:
%   - Slight positive tracking improves readability for dyslexia
%   - Headings: +20 units (microtype)
%   - Body: default (Libertinus is well-spaced)
%
% ============================================================================

% --- FONT PACKAGES ---
% Libertinus: professional serif with complete math support
% Fork of Linux Libertine with OpenType math, maintained actively
\usepackage{libertinus-type1}  % For pdfLaTeX (Type 1 fonts)

% Alternative for LuaLaTeX/XeLaTeX (uncomment if using those):
% \usepackage{libertinus-otf}

% Inconsolata for monospace (clean, accessible coding font)
\usepackage{inconsolata}

% ============================================================================
% PAGE GEOMETRY (Optimal Placement)
% ============================================================================
% Calculated for US Letter (8.5" × 11") with golden ratio principles
%
% Text block dimensions:
%   - Width: 5.25 inches (≈66 chars at 11pt Libertinus)
%   - Height: 8.0 inches (leaves room for headers/footers)
%
% Margin distribution (traditional "canonical" proportions):
%   - Inner margin: 1.0 inch (binding edge)
%   - Outer margin: 1.25 inch (thumb space + visual balance)
%   - Top margin: 1.0 inch
%   - Bottom margin: 2.0 inch (footer + page number space)
%
% Note: geometry package should be loaded BEFORE this file
% These settings override the basic 1in margins

\geometry{
    letterpaper,
    % Specify margins only; let geometry calculate text dimensions
    % This avoids over-specification warnings
    inner=1.0in,
    outer=1.25in,
    top=1.0in,
    bottom=1.5in,
    marginparwidth=0.75in,
    marginparsep=0.125in,
    headheight=24pt,          % Increased for Libertinus font height
    headsep=0.25in,
    footskip=0.5in
}

% ============================================================================
% LINE SPACING
% ============================================================================
\usepackage{setspace}
\setstretch{1.35}  % 135% line height for body text

% ============================================================================
% PARAGRAPH FORMATTING
% ============================================================================
% No paragraph indentation (block style) with vertical space between paragraphs
% This is more accessible for readers with tracking difficulties

\setlength{\parindent}{0pt}
\setlength{\parskip}{0.6\baselineskip plus 0.1\baselineskip minus 0.1\baselineskip}

% ============================================================================
% HEADING STYLES
% ============================================================================
\usepackage{titlesec}

% Section: Large, bold, with generous spacing
% Sans-serif for better visibility at larger sizes
\titleformat{\section}
    {\Large\bfseries\sffamily}  % Sans-serif bold
    {\thesection}
    {1em}
    {}
\titlespacing*{\section}
    {0pt}                        % Left margin
    {2\baselineskip plus 0.5\baselineskip}  % Before
    {1\baselineskip}             % After

% Subsection: Medium, bold
\titleformat{\subsection}
    {\large\bfseries\sffamily}
    {\thesubsection}
    {1em}
    {}
\titlespacing*{\subsection}
    {0pt}
    {1.5\baselineskip plus 0.3\baselineskip}
    {0.75\baselineskip}

% Subsubsection: Normal size, bold
\titleformat{\subsubsection}
    {\normalsize\bfseries\sffamily}
    {\thesubsubsection}
    {1em}
    {}
\titlespacing*{\subsubsection}
    {0pt}
    {1\baselineskip plus 0.2\baselineskip}
    {0.5\baselineskip}

% ============================================================================
% CAPTION STYLING
% ============================================================================
\usepackage[
    font={small,stretch=1.1},     % 10pt, slightly tighter than body
    labelfont={bf,sf},            % Bold sans-serif labels
    labelsep=period,              % "Figure 1. Caption text"
    justification=justified,
    singlelinecheck=false,
    margin=0.5in,                 % Indent captions from text margins
    format=hang,                  % Hanging indent for multi-line
    skip=0.75\baselineskip        % Space between figure and caption
]{caption}

% ============================================================================
% TABLE STYLING
% ============================================================================
\usepackage{booktabs}  % Professional table rules
\usepackage{array}     % Enhanced column types

% Increased row spacing in tables
\renewcommand{\arraystretch}{1.3}

% ============================================================================
% LIST FORMATTING
% ============================================================================
\usepackage{enumitem}

% Tighter lists with accessible spacing
\setlist{
    topsep=0.5\baselineskip,
    itemsep=0.25\baselineskip,
    parsep=0pt,
    leftmargin=1.5em
}

% Description lists (used in Appendix)
\setlist[description]{
    style=nextline,           % Label on its own line
    leftmargin=1em,           % Positive margin to avoid warning
    labelindent=0pt,
    itemsep=0.75\baselineskip,
    font=\bfseries\sffamily   % Bold sans-serif terms
}

% ============================================================================
% MATH TYPOGRAPHY
% ============================================================================
% Libertinus Math integrates seamlessly with Libertinus text

% Slightly larger math for readability
\DeclareMathSizes{11}{11}{8}{6}    % text, display, script, scriptscript

% ============================================================================
% MICROTYPE (Fine Typography)
% ============================================================================
\usepackage[
    activate={true,nocompatibility},
    final,
    tracking=true,
    factor=1100,
    stretch=10,
    shrink=10,
    nopatch=footnote  % Avoid conflict with footmisc package
]{microtype}

% Slight letter spacing for small caps (used in some abbreviations)
\SetTracking{encoding={*}, shape=sc}{40}

% ============================================================================
% HYPERLINK STYLING
% ============================================================================
% Accessible colors for links (from accessible_colors.tex palette)
\hypersetup{
    colorlinks=true,
    linkcolor={rgb:red,0;green,0.447;blue,0.698},    % PbPbcolor (deep blue)
    citecolor={rgb:red,0;green,0.447;blue,0.698},    % Same blue for citations
    urlcolor={rgb:red,0.835;green,0.369;blue,0},     % theorycolor (vermillion)
    filecolor={rgb:red,0;green,0.620;blue,0.451},    % NeNecolor (teal)
    % Note: bookmarks options moved to \usepackage[...]{hyperref} in main document
    pdfstartview=FitH
}

% ============================================================================
% HEADERS AND FOOTERS
% ============================================================================
\usepackage{fancyhdr}
\pagestyle{fancy}
\fancyhf{}  % Clear defaults

% Header: Short title (left), page number (right)
% Note: Using fixed short title instead of \leftmark to avoid overly long headers
\fancyhead[L]{\small\sffamily Light-Ion QGP: O--O \& Ne--Ne}
\fancyhead[R]{\small\sffamily\thepage}

% Footer: Empty (clean look)
\fancyfoot{}

% Line under header
\renewcommand{\headrulewidth}{0.4pt}
\renewcommand{\footrulewidth}{0pt}

% Plain style for chapter/section opening pages
\fancypagestyle{plain}{
    \fancyhf{}
    \fancyfoot[C]{\small\sffamily\thepage}
    \renewcommand{\headrulewidth}{0pt}
}

% ============================================================================
% TITLE PAGE STYLING
% ============================================================================
% Custom title format for accessibility

\makeatletter
\renewcommand{\maketitle}{%
    \begin{center}
        \vspace*{1in}
        {\LARGE\bfseries\sffamily \@title \par}
        \vspace{1.5\baselineskip}
        {\large \@author \par}
        \vspace{0.5\baselineskip}
        {\normalsize \@date \par}
        \vspace{2\baselineskip}
    \end{center}
    \thispagestyle{empty}
}
\makeatother

% ============================================================================
% FOOTNOTE STYLING
% ============================================================================
\usepackage[bottom,hang]{footmisc}
\setlength{\footnotemargin}{0.5em}
\renewcommand{\footnotelayout}{\small}

% ============================================================================
% BLOCK QUOTE STYLING
% ============================================================================
\usepackage{quoting}
\quotingsetup{
    leftmargin=1.5em,
    rightmargin=1.5em,
    vskip=0.5\baselineskip,
    font={small,itshape}
}

% ============================================================================
% ACCESSIBILITY NOTES
% ============================================================================
%
% FONT CHOICE RATIONALE:
%   - Libertinus is a high-quality open-source serif with:
%     * Clear letterforms with good x-height
%     * Distinct character shapes (reduces confusion: l/1/I, O/0)
%     * Complete math support (Libertinus Math)
%     * Active maintenance and bug fixes
%
% SANS-SERIF FOR HEADINGS:
%   - Research shows sans-serif is clearer at larger sizes
%   - Libertinus Sans matches Libertinus Serif metrics
%   - Creates clear visual hierarchy
%
% LINE SPACING RATIONALE:
%   - 1.35× (135%) exceeds minimum 1.5× recommended by some guidelines
%   - but 1.5× was found excessive for academic text
%   - 1.35× balances readability with page economy
%
% AVOIDING PROBLEMATIC ELEMENTS:
%   - No justified text with hyphenation (causes "rivers")
%   - No pure black text on pure white (too high contrast)
%   - Minimal italics (hard for dyslexic readers)
%   - No all-caps headings (reduces recognition)
%
% ============================================================================
