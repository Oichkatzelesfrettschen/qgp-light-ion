% flow_visualization.tex
%
% Defines the PGFPlot for the anisotropic flow coefficients (v_n) vs. collision centrality.
% This visualization illustrates how the initial geometry of the collision system
% (e.g., O-O vs. Ne-Ne) and event-by-event fluctuations are translated into
% momentum-space anisotropies of the final-state particles.

\documentclass{standalone}
\usepackage{pgfplots}
\pgfplotsset{compat=1.18}

\begin{document}
\begin{tikzpicture}
    % The 'axis' environment. Note the 'x dir=reverse' option, which is a standard
    % convention in heavy-ion physics to place the most central (0%) collisions
    % on the right-hand side of the x-axis.
    \begin{axis}[
        title={Anisotropic Flow ($v_n$) vs. Centrality},
        xlabel={Centrality \%},
        ylabel={$v_n$ coefficients},
        xmin=-2, xmax=82,
        ymin=0, ymax=0.09,
        x dir=reverse,
        legend pos=north west,
        grid=both,
        grid style={line width=.1pt, draw=gray!30},
        major grid style={line width=.2pt,draw=gray!50},
        minor tick num=1,
        width=15cm,
        height=9cm,
        % Use the same cycle list as other plots for a consistent, professional theme.
        cycle list name=exotic,
    ]

    % Plot 1: v2 for O-O collisions. This is the elliptic flow, primarily driven
    % by the almond-shaped overlap geometry in non-central collisions.
    \addplot[
        thick,
        mark=square*,
    ] table[x index=0, y index=1] {data/flow_v2_OO.dat};
    \addlegendentry{$v_2$ (O--O)}

    % Plot 2: v3 for O-O collisions. This is the triangular flow, primarily driven
    % by event-by-event fluctuations in the initial nucleon positions.
    \addplot[
        thick,
        mark=triangle*,
    ] table[x index=0, y index=1] {data/flow_v3_OO.dat};
    \addlegendentry{$v_3$ (O--O)}
    
    % Plot 3: v2 for Ne-Ne. This is plotted with a dashed line to indicate a comparison
    % to the primary O-O system. The higher value at low centrality demonstrates the
    % effect of Neon's intrinsic prolate deformation on the initial geometry.
    \addplot[
        dashed, % Dashed line for comparison/systematic variation.
        thick,
        mark=diamond*,
    ] table[x index=0, y index=1] {data/flow_v2_NeNe.dat};
    \addlegendentry{$v_2$ (Ne--Ne, prolate)}

    \end{axis}
\end{tikzpicture}
\end{document}