% RAA_visualization.tex
%
% Defines the PGFPlot for the Nuclear Modification Factor (R_AA) vs. Transverse Momentum (p_T).
% This visualization compares the jet quenching effect in O-O and Pb-Pb collisions.
% Data is loaded from external files generated by generate_plot_data.py.

\documentclass[border=3mm]{standalone}
\usepackage{pgfplots}
\pgfplotsset{compat=1.18} % Sets compatibility mode for stable behavior.

\begin{document}
\begin{tikzpicture}
    % The 'axis' environment defines the coordinate system and plot elements.
    % Each key-value pair is set for optimal clarity and adherence to scientific plotting standards.
    \begin{axis}[
        title={Nuclear Modification Factor ($R_{AA}$) vs. Transverse Momentum ($p_T$)},
        xlabel={$p_T$ (GeV/$c$)},          % x-axis label with standard units.
        ylabel={$R_{AA}$},                % y-axis label.
        xmin=0, xmax=22,                  % Axis limits are set slightly beyond data range.
        ymin=0, ymax=1.2,
        xtick distance=2,                 % Explicitly define major tick spacing.
        minor tick num=1,                 % Adds one minor tick between major ticks for granularity.
        legend pos=south east,            % Legend positioned to avoid data occlusion.
        grid=both,                        % Display both major and minor grid lines.
        grid style={line width=.1pt, draw=gray!30},
        major grid style={line width=.2pt,draw=gray!50},
        width=15cm,                       % Set plot dimensions.
        height=9cm,
        % Define a cycle list for plot styles. This provides a professional and
        % consistent set of colors and markers for each data series.
        cycle list name=exotic,
    ]

    % A dashed line at R_AA = 1 serves as the baseline reference for no nuclear modification.
    \addplot[black, dashed, domain=0:22, samples=2] {1} node[pos=0.1, above right, font=\small] {$R_{AA}=1$};

    % Plot 1: O-O data.
    % Uses 'addplot table' to load data from a file, separating data from presentation.
    % Column indices are specified for x, y, and y-error values.
    \addplot[
        thick, % Use a thicker line for better visibility.
        error bars/.cd,
        y dir=both, y explicit,
    ] table[
        x index=0, y index=1, y error index=2
    ] {data/RAA_OO.dat};
    % The \addlegendentry command must follow the corresponding \addplot.
    \addlegendentry{O--O at $\sqrt{s_{NN}} = 5.36$ TeV}

    % Plot 2: Pb-Pb data. This plot automatically uses the second style from the cycle list.
    \addplot[
        thick,
        error bars/.cd,
        y dir=both, y explicit,
    ] table[
        x index=0, y index=1, y error index=2
    ] {data/RAA_PbPb.dat};
    \addlegendentry{Pb--Pb at $\sqrt{s_{NN}} = 5.02$ TeV}

    \end{axis}
\end{tikzpicture}
\end{document}