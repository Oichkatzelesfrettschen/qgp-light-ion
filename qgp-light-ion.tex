\documentclass[11pt]{article}

% ============================================================================
% CORE PACKAGES (Load order matters)
% ============================================================================

% Unicode input support (for Pandoc-generated content with Greek letters, etc.)
\usepackage[utf8]{inputenc}
\usepackage[T1]{fontenc}
\usepackage{textcomp}     % Additional text symbols (≈, √, etc.)
\usepackage{textgreek}    % Greek letters in text mode

% Geometry first (will be reconfigured by typography file)
\usepackage[margin=1in]{geometry}

% Graphics and math
\usepackage{graphicx}
\usepackage{amsmath}
\usepackage{amssymb}

% Tables (longtable for Pandoc-generated tables, booktabs for professional formatting)
\usepackage{longtable}
\usepackage{booktabs}
\usepackage{array}        % Enhanced column specifications
\usepackage{calc}         % For \real{} in Pandoc table widths

% Define 'none' counter for Pandoc's \def\LTcaptype{none} construct
\newcounter{none}

% Define \real{} command for Pandoc table column widths
% Pandoc uses: p{(\linewidth - 8\tabcolsep) * \real{0.XXXX}}
% With calc package, \real{} should just pass through the decimal factor
\providecommand{\real}[1]{#1}

% Fix "Infinite glue shrinkage" errors in longtable page breaks
% raggedbottom allows flexible page bottoms instead of forcing \flushbottom
\raggedbottom

% Suppress infinite shrink errors from longtable (TeX can recover gracefully)
% These occur when tables span page boundaries with stretchable glue
\makeatletter
\def\@textbottom{\vskip \z@ \@plus 1pt}
\let\@texttop\relax
\makeatother

% Colors (before hyperref)
\usepackage{xcolor}

% Hyperlinks (before natbib for compatibility)
% bookmarks options must be set at load time, not via hypersetup
\usepackage[bookmarks=true,bookmarksnumbered=true]{hyperref}
% bookmark package: eliminates "rerunfilecheck" warnings by handling bookmarks more efficiently
\usepackage{bookmark}

% Bibliography - superscript citations
\usepackage[super,sort&compress]{natbib}

% ============================================================================
% ACCESSIBLE TYPOGRAPHY CONFIGURATION
% ============================================================================
% Libertinus fonts + optimal layout + accessibility features
% See figures/accessible_typography.tex for detailed rationale
% accessible_typography.tex
%
% ACCESSIBLE TYPOGRAPHY CONFIGURATION FOR QGP LIGHT-ION PROJECT
% Optimized for:
%   - Readability across visual abilities (dyslexia, low vision)
%   - Scientific document conventions
%   - Alignment with accessible_colors.tex color scheme
%   - Print and screen compatibility
%
% Based on: Libertinus font family + typographic research best practices
%
% USAGE: % accessible_typography.tex
%
% ACCESSIBLE TYPOGRAPHY CONFIGURATION FOR QGP LIGHT-ION PROJECT
% Optimized for:
%   - Readability across visual abilities (dyslexia, low vision)
%   - Scientific document conventions
%   - Alignment with accessible_colors.tex color scheme
%   - Print and screen compatibility
%
% Based on: Libertinus font family + typographic research best practices
%
% USAGE: % accessible_typography.tex
%
% ACCESSIBLE TYPOGRAPHY CONFIGURATION FOR QGP LIGHT-ION PROJECT
% Optimized for:
%   - Readability across visual abilities (dyslexia, low vision)
%   - Scientific document conventions
%   - Alignment with accessible_colors.tex color scheme
%   - Print and screen compatibility
%
% Based on: Libertinus font family + typographic research best practices
%
% USAGE: \input{figures/accessible_typography.tex} in preamble AFTER geometry
%
% ============================================================================
% TYPOGRAPHIC RATIONALE
% ============================================================================
%
% LINE LENGTH (Measure):
%   - Optimal: 60-75 characters per line (research consensus)
%   - At 11pt Libertinus, ~66 chars ≈ 5.0 inches text width
%   - Wider margins improve focus and allow marginal notes
%
% LINE HEIGHT (Leading):
%   - Body text: 1.35× (130-140% recommended for accessibility)
%   - Captions: 1.25× (tighter for shorter text blocks)
%   - Code/tables: 1.15× (dense information)
%
% FONT SIZE HIERARCHY:
%   - Body: 11pt (standard academic, good for extended reading)
%   - Captions: 10pt (0.9× body)
%   - Footnotes: 9pt (0.8× body)
%   - Section headings: 14pt bold (1.27× body)
%   - Subsection: 12pt bold (1.09× body)
%
% LETTER SPACING:
%   - Slight positive tracking improves readability for dyslexia
%   - Headings: +20 units (microtype)
%   - Body: default (Libertinus is well-spaced)
%
% ============================================================================

% --- FONT PACKAGES ---
% Libertinus: professional serif with complete math support
% Fork of Linux Libertine with OpenType math, maintained actively
\usepackage{libertinus-type1}  % For pdfLaTeX (Type 1 fonts)

% Alternative for LuaLaTeX/XeLaTeX (uncomment if using those):
% \usepackage{libertinus-otf}

% Inconsolata for monospace (clean, accessible coding font)
\usepackage{inconsolata}

% ============================================================================
% PAGE GEOMETRY (Optimal Placement)
% ============================================================================
% Calculated for US Letter (8.5" × 11") with golden ratio principles
%
% Text block dimensions:
%   - Width: 5.25 inches (≈66 chars at 11pt Libertinus)
%   - Height: 8.0 inches (leaves room for headers/footers)
%
% Margin distribution (traditional "canonical" proportions):
%   - Inner margin: 1.0 inch (binding edge)
%   - Outer margin: 1.25 inch (thumb space + visual balance)
%   - Top margin: 1.0 inch
%   - Bottom margin: 2.0 inch (footer + page number space)
%
% Note: geometry package should be loaded BEFORE this file
% These settings override the basic 1in margins

\geometry{
    letterpaper,
    % Specify margins only; let geometry calculate text dimensions
    % This avoids over-specification warnings
    inner=1.0in,
    outer=1.25in,
    top=1.0in,
    bottom=1.5in,
    marginparwidth=0.75in,
    marginparsep=0.125in,
    headheight=24pt,          % Increased for Libertinus font height
    headsep=0.25in,
    footskip=0.5in
}

% ============================================================================
% LINE SPACING
% ============================================================================
\usepackage{setspace}
\setstretch{1.35}  % 135% line height for body text

% ============================================================================
% PARAGRAPH FORMATTING
% ============================================================================
% No paragraph indentation (block style) with vertical space between paragraphs
% This is more accessible for readers with tracking difficulties

\setlength{\parindent}{0pt}
\setlength{\parskip}{0.6\baselineskip plus 0.1\baselineskip minus 0.1\baselineskip}

% ============================================================================
% HEADING STYLES
% ============================================================================
\usepackage{titlesec}

% Section: Large, bold, with generous spacing
% Sans-serif for better visibility at larger sizes
\titleformat{\section}
    {\Large\bfseries\sffamily}  % Sans-serif bold
    {\thesection}
    {1em}
    {}
\titlespacing*{\section}
    {0pt}                        % Left margin
    {2\baselineskip plus 0.5\baselineskip}  % Before
    {1\baselineskip}             % After

% Subsection: Medium, bold
\titleformat{\subsection}
    {\large\bfseries\sffamily}
    {\thesubsection}
    {1em}
    {}
\titlespacing*{\subsection}
    {0pt}
    {1.5\baselineskip plus 0.3\baselineskip}
    {0.75\baselineskip}

% Subsubsection: Normal size, bold
\titleformat{\subsubsection}
    {\normalsize\bfseries\sffamily}
    {\thesubsubsection}
    {1em}
    {}
\titlespacing*{\subsubsection}
    {0pt}
    {1\baselineskip plus 0.2\baselineskip}
    {0.5\baselineskip}

% ============================================================================
% CAPTION STYLING
% ============================================================================
\usepackage[
    font={small,stretch=1.1},     % 10pt, slightly tighter than body
    labelfont={bf,sf},            % Bold sans-serif labels
    labelsep=period,              % "Figure 1. Caption text"
    justification=justified,
    singlelinecheck=false,
    margin=0.5in,                 % Indent captions from text margins
    format=hang,                  % Hanging indent for multi-line
    skip=0.75\baselineskip        % Space between figure and caption
]{caption}

% ============================================================================
% TABLE STYLING
% ============================================================================
\usepackage{booktabs}  % Professional table rules
\usepackage{array}     % Enhanced column types

% Increased row spacing in tables
\renewcommand{\arraystretch}{1.3}

% ============================================================================
% LIST FORMATTING
% ============================================================================
\usepackage{enumitem}

% Tighter lists with accessible spacing
\setlist{
    topsep=0.5\baselineskip,
    itemsep=0.25\baselineskip,
    parsep=0pt,
    leftmargin=1.5em
}

% Description lists (used in Appendix)
\setlist[description]{
    style=nextline,           % Label on its own line
    leftmargin=1em,           % Positive margin to avoid warning
    labelindent=0pt,
    itemsep=0.75\baselineskip,
    font=\bfseries\sffamily   % Bold sans-serif terms
}

% ============================================================================
% MATH TYPOGRAPHY
% ============================================================================
% Libertinus Math integrates seamlessly with Libertinus text

% Slightly larger math for readability
\DeclareMathSizes{11}{11}{8}{6}    % text, display, script, scriptscript

% ============================================================================
% MICROTYPE (Fine Typography)
% ============================================================================
\usepackage[
    activate={true,nocompatibility},
    final,
    tracking=true,
    factor=1100,
    stretch=10,
    shrink=10,
    nopatch=footnote  % Avoid conflict with footmisc package
]{microtype}

% Slight letter spacing for small caps (used in some abbreviations)
\SetTracking{encoding={*}, shape=sc}{40}

% ============================================================================
% HYPERLINK STYLING
% ============================================================================
% Accessible colors for links (from accessible_colors.tex palette)
\hypersetup{
    colorlinks=true,
    linkcolor={rgb:red,0;green,0.447;blue,0.698},    % PbPbcolor (deep blue)
    citecolor={rgb:red,0;green,0.447;blue,0.698},    % Same blue for citations
    urlcolor={rgb:red,0.835;green,0.369;blue,0},     % theorycolor (vermillion)
    filecolor={rgb:red,0;green,0.620;blue,0.451},    % NeNecolor (teal)
    % Note: bookmarks options moved to \usepackage[...]{hyperref} in main document
    pdfstartview=FitH
}

% ============================================================================
% HEADERS AND FOOTERS
% ============================================================================
\usepackage{fancyhdr}
\pagestyle{fancy}
\fancyhf{}  % Clear defaults

% Header: Short title (left), page number (right)
% Note: Using fixed short title instead of \leftmark to avoid overly long headers
\fancyhead[L]{\small\sffamily Light-Ion QGP: O--O \& Ne--Ne}
\fancyhead[R]{\small\sffamily\thepage}

% Footer: Empty (clean look)
\fancyfoot{}

% Line under header
\renewcommand{\headrulewidth}{0.4pt}
\renewcommand{\footrulewidth}{0pt}

% Plain style for chapter/section opening pages
\fancypagestyle{plain}{
    \fancyhf{}
    \fancyfoot[C]{\small\sffamily\thepage}
    \renewcommand{\headrulewidth}{0pt}
}

% ============================================================================
% TITLE PAGE STYLING
% ============================================================================
% Custom title format for accessibility

\makeatletter
\renewcommand{\maketitle}{%
    \begin{center}
        \vspace*{1in}
        {\LARGE\bfseries\sffamily \@title \par}
        \vspace{1.5\baselineskip}
        {\large \@author \par}
        \vspace{0.5\baselineskip}
        {\normalsize \@date \par}
        \vspace{2\baselineskip}
    \end{center}
    \thispagestyle{empty}
}
\makeatother

% ============================================================================
% FOOTNOTE STYLING
% ============================================================================
\usepackage[bottom,hang]{footmisc}
\setlength{\footnotemargin}{0.5em}
\renewcommand{\footnotelayout}{\small}

% ============================================================================
% BLOCK QUOTE STYLING
% ============================================================================
\usepackage{quoting}
\quotingsetup{
    leftmargin=1.5em,
    rightmargin=1.5em,
    vskip=0.5\baselineskip,
    font={small,itshape}
}

% ============================================================================
% ACCESSIBILITY NOTES
% ============================================================================
%
% FONT CHOICE RATIONALE:
%   - Libertinus is a high-quality open-source serif with:
%     * Clear letterforms with good x-height
%     * Distinct character shapes (reduces confusion: l/1/I, O/0)
%     * Complete math support (Libertinus Math)
%     * Active maintenance and bug fixes
%
% SANS-SERIF FOR HEADINGS:
%   - Research shows sans-serif is clearer at larger sizes
%   - Libertinus Sans matches Libertinus Serif metrics
%   - Creates clear visual hierarchy
%
% LINE SPACING RATIONALE:
%   - 1.35× (135%) exceeds minimum 1.5× recommended by some guidelines
%   - but 1.5× was found excessive for academic text
%   - 1.35× balances readability with page economy
%
% AVOIDING PROBLEMATIC ELEMENTS:
%   - No justified text with hyphenation (causes "rivers")
%   - No pure black text on pure white (too high contrast)
%   - Minimal italics (hard for dyslexic readers)
%   - No all-caps headings (reduces recognition)
%
% ============================================================================
 in preamble AFTER geometry
%
% ============================================================================
% TYPOGRAPHIC RATIONALE
% ============================================================================
%
% LINE LENGTH (Measure):
%   - Optimal: 60-75 characters per line (research consensus)
%   - At 11pt Libertinus, ~66 chars ≈ 5.0 inches text width
%   - Wider margins improve focus and allow marginal notes
%
% LINE HEIGHT (Leading):
%   - Body text: 1.35× (130-140% recommended for accessibility)
%   - Captions: 1.25× (tighter for shorter text blocks)
%   - Code/tables: 1.15× (dense information)
%
% FONT SIZE HIERARCHY:
%   - Body: 11pt (standard academic, good for extended reading)
%   - Captions: 10pt (0.9× body)
%   - Footnotes: 9pt (0.8× body)
%   - Section headings: 14pt bold (1.27× body)
%   - Subsection: 12pt bold (1.09× body)
%
% LETTER SPACING:
%   - Slight positive tracking improves readability for dyslexia
%   - Headings: +20 units (microtype)
%   - Body: default (Libertinus is well-spaced)
%
% ============================================================================

% --- FONT PACKAGES ---
% Libertinus: professional serif with complete math support
% Fork of Linux Libertine with OpenType math, maintained actively
\usepackage{libertinus-type1}  % For pdfLaTeX (Type 1 fonts)

% Alternative for LuaLaTeX/XeLaTeX (uncomment if using those):
% \usepackage{libertinus-otf}

% Inconsolata for monospace (clean, accessible coding font)
\usepackage{inconsolata}

% ============================================================================
% PAGE GEOMETRY (Optimal Placement)
% ============================================================================
% Calculated for US Letter (8.5" × 11") with golden ratio principles
%
% Text block dimensions:
%   - Width: 5.25 inches (≈66 chars at 11pt Libertinus)
%   - Height: 8.0 inches (leaves room for headers/footers)
%
% Margin distribution (traditional "canonical" proportions):
%   - Inner margin: 1.0 inch (binding edge)
%   - Outer margin: 1.25 inch (thumb space + visual balance)
%   - Top margin: 1.0 inch
%   - Bottom margin: 2.0 inch (footer + page number space)
%
% Note: geometry package should be loaded BEFORE this file
% These settings override the basic 1in margins

\geometry{
    letterpaper,
    % Specify margins only; let geometry calculate text dimensions
    % This avoids over-specification warnings
    inner=1.0in,
    outer=1.25in,
    top=1.0in,
    bottom=1.5in,
    marginparwidth=0.75in,
    marginparsep=0.125in,
    headheight=24pt,          % Increased for Libertinus font height
    headsep=0.25in,
    footskip=0.5in
}

% ============================================================================
% LINE SPACING
% ============================================================================
\usepackage{setspace}
\setstretch{1.35}  % 135% line height for body text

% ============================================================================
% PARAGRAPH FORMATTING
% ============================================================================
% No paragraph indentation (block style) with vertical space between paragraphs
% This is more accessible for readers with tracking difficulties

\setlength{\parindent}{0pt}
\setlength{\parskip}{0.6\baselineskip plus 0.1\baselineskip minus 0.1\baselineskip}

% ============================================================================
% HEADING STYLES
% ============================================================================
\usepackage{titlesec}

% Section: Large, bold, with generous spacing
% Sans-serif for better visibility at larger sizes
\titleformat{\section}
    {\Large\bfseries\sffamily}  % Sans-serif bold
    {\thesection}
    {1em}
    {}
\titlespacing*{\section}
    {0pt}                        % Left margin
    {2\baselineskip plus 0.5\baselineskip}  % Before
    {1\baselineskip}             % After

% Subsection: Medium, bold
\titleformat{\subsection}
    {\large\bfseries\sffamily}
    {\thesubsection}
    {1em}
    {}
\titlespacing*{\subsection}
    {0pt}
    {1.5\baselineskip plus 0.3\baselineskip}
    {0.75\baselineskip}

% Subsubsection: Normal size, bold
\titleformat{\subsubsection}
    {\normalsize\bfseries\sffamily}
    {\thesubsubsection}
    {1em}
    {}
\titlespacing*{\subsubsection}
    {0pt}
    {1\baselineskip plus 0.2\baselineskip}
    {0.5\baselineskip}

% ============================================================================
% CAPTION STYLING
% ============================================================================
\usepackage[
    font={small,stretch=1.1},     % 10pt, slightly tighter than body
    labelfont={bf,sf},            % Bold sans-serif labels
    labelsep=period,              % "Figure 1. Caption text"
    justification=justified,
    singlelinecheck=false,
    margin=0.5in,                 % Indent captions from text margins
    format=hang,                  % Hanging indent for multi-line
    skip=0.75\baselineskip        % Space between figure and caption
]{caption}

% ============================================================================
% TABLE STYLING
% ============================================================================
\usepackage{booktabs}  % Professional table rules
\usepackage{array}     % Enhanced column types

% Increased row spacing in tables
\renewcommand{\arraystretch}{1.3}

% ============================================================================
% LIST FORMATTING
% ============================================================================
\usepackage{enumitem}

% Tighter lists with accessible spacing
\setlist{
    topsep=0.5\baselineskip,
    itemsep=0.25\baselineskip,
    parsep=0pt,
    leftmargin=1.5em
}

% Description lists (used in Appendix)
\setlist[description]{
    style=nextline,           % Label on its own line
    leftmargin=1em,           % Positive margin to avoid warning
    labelindent=0pt,
    itemsep=0.75\baselineskip,
    font=\bfseries\sffamily   % Bold sans-serif terms
}

% ============================================================================
% MATH TYPOGRAPHY
% ============================================================================
% Libertinus Math integrates seamlessly with Libertinus text

% Slightly larger math for readability
\DeclareMathSizes{11}{11}{8}{6}    % text, display, script, scriptscript

% ============================================================================
% MICROTYPE (Fine Typography)
% ============================================================================
\usepackage[
    activate={true,nocompatibility},
    final,
    tracking=true,
    factor=1100,
    stretch=10,
    shrink=10,
    nopatch=footnote  % Avoid conflict with footmisc package
]{microtype}

% Slight letter spacing for small caps (used in some abbreviations)
\SetTracking{encoding={*}, shape=sc}{40}

% ============================================================================
% HYPERLINK STYLING
% ============================================================================
% Accessible colors for links (from accessible_colors.tex palette)
\hypersetup{
    colorlinks=true,
    linkcolor={rgb:red,0;green,0.447;blue,0.698},    % PbPbcolor (deep blue)
    citecolor={rgb:red,0;green,0.447;blue,0.698},    % Same blue for citations
    urlcolor={rgb:red,0.835;green,0.369;blue,0},     % theorycolor (vermillion)
    filecolor={rgb:red,0;green,0.620;blue,0.451},    % NeNecolor (teal)
    % Note: bookmarks options moved to \usepackage[...]{hyperref} in main document
    pdfstartview=FitH
}

% ============================================================================
% HEADERS AND FOOTERS
% ============================================================================
\usepackage{fancyhdr}
\pagestyle{fancy}
\fancyhf{}  % Clear defaults

% Header: Short title (left), page number (right)
% Note: Using fixed short title instead of \leftmark to avoid overly long headers
\fancyhead[L]{\small\sffamily Light-Ion QGP: O--O \& Ne--Ne}
\fancyhead[R]{\small\sffamily\thepage}

% Footer: Empty (clean look)
\fancyfoot{}

% Line under header
\renewcommand{\headrulewidth}{0.4pt}
\renewcommand{\footrulewidth}{0pt}

% Plain style for chapter/section opening pages
\fancypagestyle{plain}{
    \fancyhf{}
    \fancyfoot[C]{\small\sffamily\thepage}
    \renewcommand{\headrulewidth}{0pt}
}

% ============================================================================
% TITLE PAGE STYLING
% ============================================================================
% Custom title format for accessibility

\makeatletter
\renewcommand{\maketitle}{%
    \begin{center}
        \vspace*{1in}
        {\LARGE\bfseries\sffamily \@title \par}
        \vspace{1.5\baselineskip}
        {\large \@author \par}
        \vspace{0.5\baselineskip}
        {\normalsize \@date \par}
        \vspace{2\baselineskip}
    \end{center}
    \thispagestyle{empty}
}
\makeatother

% ============================================================================
% FOOTNOTE STYLING
% ============================================================================
\usepackage[bottom,hang]{footmisc}
\setlength{\footnotemargin}{0.5em}
\renewcommand{\footnotelayout}{\small}

% ============================================================================
% BLOCK QUOTE STYLING
% ============================================================================
\usepackage{quoting}
\quotingsetup{
    leftmargin=1.5em,
    rightmargin=1.5em,
    vskip=0.5\baselineskip,
    font={small,itshape}
}

% ============================================================================
% ACCESSIBILITY NOTES
% ============================================================================
%
% FONT CHOICE RATIONALE:
%   - Libertinus is a high-quality open-source serif with:
%     * Clear letterforms with good x-height
%     * Distinct character shapes (reduces confusion: l/1/I, O/0)
%     * Complete math support (Libertinus Math)
%     * Active maintenance and bug fixes
%
% SANS-SERIF FOR HEADINGS:
%   - Research shows sans-serif is clearer at larger sizes
%   - Libertinus Sans matches Libertinus Serif metrics
%   - Creates clear visual hierarchy
%
% LINE SPACING RATIONALE:
%   - 1.35× (135%) exceeds minimum 1.5× recommended by some guidelines
%   - but 1.5× was found excessive for academic text
%   - 1.35× balances readability with page economy
%
% AVOIDING PROBLEMATIC ELEMENTS:
%   - No justified text with hyphenation (causes "rivers")
%   - No pure black text on pure white (too high contrast)
%   - Minimal italics (hard for dyslexic readers)
%   - No all-caps headings (reduces recognition)
%
% ============================================================================
 in preamble AFTER geometry
%
% ============================================================================
% TYPOGRAPHIC RATIONALE
% ============================================================================
%
% LINE LENGTH (Measure):
%   - Optimal: 60-75 characters per line (research consensus)
%   - At 11pt Libertinus, ~66 chars ≈ 5.0 inches text width
%   - Wider margins improve focus and allow marginal notes
%
% LINE HEIGHT (Leading):
%   - Body text: 1.35× (130-140% recommended for accessibility)
%   - Captions: 1.25× (tighter for shorter text blocks)
%   - Code/tables: 1.15× (dense information)
%
% FONT SIZE HIERARCHY:
%   - Body: 11pt (standard academic, good for extended reading)
%   - Captions: 10pt (0.9× body)
%   - Footnotes: 9pt (0.8× body)
%   - Section headings: 14pt bold (1.27× body)
%   - Subsection: 12pt bold (1.09× body)
%
% LETTER SPACING:
%   - Slight positive tracking improves readability for dyslexia
%   - Headings: +20 units (microtype)
%   - Body: default (Libertinus is well-spaced)
%
% ============================================================================

% --- FONT PACKAGES ---
% Libertinus: professional serif with complete math support
% Fork of Linux Libertine with OpenType math, maintained actively
\usepackage{libertinus-type1}  % For pdfLaTeX (Type 1 fonts)

% Alternative for LuaLaTeX/XeLaTeX (uncomment if using those):
% \usepackage{libertinus-otf}

% Inconsolata for monospace (clean, accessible coding font)
\usepackage{inconsolata}

% ============================================================================
% PAGE GEOMETRY (Optimal Placement)
% ============================================================================
% Calculated for US Letter (8.5" × 11") with golden ratio principles
%
% Text block dimensions:
%   - Width: 5.25 inches (≈66 chars at 11pt Libertinus)
%   - Height: 8.0 inches (leaves room for headers/footers)
%
% Margin distribution (traditional "canonical" proportions):
%   - Inner margin: 1.0 inch (binding edge)
%   - Outer margin: 1.25 inch (thumb space + visual balance)
%   - Top margin: 1.0 inch
%   - Bottom margin: 2.0 inch (footer + page number space)
%
% Note: geometry package should be loaded BEFORE this file
% These settings override the basic 1in margins

\geometry{
    letterpaper,
    % Specify margins only; let geometry calculate text dimensions
    % This avoids over-specification warnings
    inner=1.0in,
    outer=1.25in,
    top=1.0in,
    bottom=1.5in,
    marginparwidth=0.75in,
    marginparsep=0.125in,
    headheight=24pt,          % Increased for Libertinus font height
    headsep=0.25in,
    footskip=0.5in
}

% ============================================================================
% LINE SPACING
% ============================================================================
\usepackage{setspace}
\setstretch{1.35}  % 135% line height for body text

% ============================================================================
% PARAGRAPH FORMATTING
% ============================================================================
% No paragraph indentation (block style) with vertical space between paragraphs
% This is more accessible for readers with tracking difficulties

\setlength{\parindent}{0pt}
\setlength{\parskip}{0.6\baselineskip plus 0.1\baselineskip minus 0.1\baselineskip}

% ============================================================================
% HEADING STYLES
% ============================================================================
\usepackage{titlesec}

% Section: Large, bold, with generous spacing
% Sans-serif for better visibility at larger sizes
\titleformat{\section}
    {\Large\bfseries\sffamily}  % Sans-serif bold
    {\thesection}
    {1em}
    {}
\titlespacing*{\section}
    {0pt}                        % Left margin
    {2\baselineskip plus 0.5\baselineskip}  % Before
    {1\baselineskip}             % After

% Subsection: Medium, bold
\titleformat{\subsection}
    {\large\bfseries\sffamily}
    {\thesubsection}
    {1em}
    {}
\titlespacing*{\subsection}
    {0pt}
    {1.5\baselineskip plus 0.3\baselineskip}
    {0.75\baselineskip}

% Subsubsection: Normal size, bold
\titleformat{\subsubsection}
    {\normalsize\bfseries\sffamily}
    {\thesubsubsection}
    {1em}
    {}
\titlespacing*{\subsubsection}
    {0pt}
    {1\baselineskip plus 0.2\baselineskip}
    {0.5\baselineskip}

% ============================================================================
% CAPTION STYLING
% ============================================================================
\usepackage[
    font={small,stretch=1.1},     % 10pt, slightly tighter than body
    labelfont={bf,sf},            % Bold sans-serif labels
    labelsep=period,              % "Figure 1. Caption text"
    justification=justified,
    singlelinecheck=false,
    margin=0.5in,                 % Indent captions from text margins
    format=hang,                  % Hanging indent for multi-line
    skip=0.75\baselineskip        % Space between figure and caption
]{caption}

% ============================================================================
% TABLE STYLING
% ============================================================================
\usepackage{booktabs}  % Professional table rules
\usepackage{array}     % Enhanced column types

% Increased row spacing in tables
\renewcommand{\arraystretch}{1.3}

% ============================================================================
% LIST FORMATTING
% ============================================================================
\usepackage{enumitem}

% Tighter lists with accessible spacing
\setlist{
    topsep=0.5\baselineskip,
    itemsep=0.25\baselineskip,
    parsep=0pt,
    leftmargin=1.5em
}

% Description lists (used in Appendix)
\setlist[description]{
    style=nextline,           % Label on its own line
    leftmargin=1em,           % Positive margin to avoid warning
    labelindent=0pt,
    itemsep=0.75\baselineskip,
    font=\bfseries\sffamily   % Bold sans-serif terms
}

% ============================================================================
% MATH TYPOGRAPHY
% ============================================================================
% Libertinus Math integrates seamlessly with Libertinus text

% Slightly larger math for readability
\DeclareMathSizes{11}{11}{8}{6}    % text, display, script, scriptscript

% ============================================================================
% MICROTYPE (Fine Typography)
% ============================================================================
\usepackage[
    activate={true,nocompatibility},
    final,
    tracking=true,
    factor=1100,
    stretch=10,
    shrink=10,
    nopatch=footnote  % Avoid conflict with footmisc package
]{microtype}

% Slight letter spacing for small caps (used in some abbreviations)
\SetTracking{encoding={*}, shape=sc}{40}

% ============================================================================
% HYPERLINK STYLING
% ============================================================================
% Accessible colors for links (from accessible_colors.tex palette)
\hypersetup{
    colorlinks=true,
    linkcolor={rgb:red,0;green,0.447;blue,0.698},    % PbPbcolor (deep blue)
    citecolor={rgb:red,0;green,0.447;blue,0.698},    % Same blue for citations
    urlcolor={rgb:red,0.835;green,0.369;blue,0},     % theorycolor (vermillion)
    filecolor={rgb:red,0;green,0.620;blue,0.451},    % NeNecolor (teal)
    % Note: bookmarks options moved to \usepackage[...]{hyperref} in main document
    pdfstartview=FitH
}

% ============================================================================
% HEADERS AND FOOTERS
% ============================================================================
\usepackage{fancyhdr}
\pagestyle{fancy}
\fancyhf{}  % Clear defaults

% Header: Short title (left), page number (right)
% Note: Using fixed short title instead of \leftmark to avoid overly long headers
\fancyhead[L]{\small\sffamily Light-Ion QGP: O--O \& Ne--Ne}
\fancyhead[R]{\small\sffamily\thepage}

% Footer: Empty (clean look)
\fancyfoot{}

% Line under header
\renewcommand{\headrulewidth}{0.4pt}
\renewcommand{\footrulewidth}{0pt}

% Plain style for chapter/section opening pages
\fancypagestyle{plain}{
    \fancyhf{}
    \fancyfoot[C]{\small\sffamily\thepage}
    \renewcommand{\headrulewidth}{0pt}
}

% ============================================================================
% TITLE PAGE STYLING
% ============================================================================
% Custom title format for accessibility

\makeatletter
\renewcommand{\maketitle}{%
    \begin{center}
        \vspace*{1in}
        {\LARGE\bfseries\sffamily \@title \par}
        \vspace{1.5\baselineskip}
        {\large \@author \par}
        \vspace{0.5\baselineskip}
        {\normalsize \@date \par}
        \vspace{2\baselineskip}
    \end{center}
    \thispagestyle{empty}
}
\makeatother

% ============================================================================
% FOOTNOTE STYLING
% ============================================================================
\usepackage[bottom,hang]{footmisc}
\setlength{\footnotemargin}{0.5em}
\renewcommand{\footnotelayout}{\small}

% ============================================================================
% BLOCK QUOTE STYLING
% ============================================================================
\usepackage{quoting}
\quotingsetup{
    leftmargin=1.5em,
    rightmargin=1.5em,
    vskip=0.5\baselineskip,
    font={small,itshape}
}

% ============================================================================
% ACCESSIBILITY NOTES
% ============================================================================
%
% FONT CHOICE RATIONALE:
%   - Libertinus is a high-quality open-source serif with:
%     * Clear letterforms with good x-height
%     * Distinct character shapes (reduces confusion: l/1/I, O/0)
%     * Complete math support (Libertinus Math)
%     * Active maintenance and bug fixes
%
% SANS-SERIF FOR HEADINGS:
%   - Research shows sans-serif is clearer at larger sizes
%   - Libertinus Sans matches Libertinus Serif metrics
%   - Creates clear visual hierarchy
%
% LINE SPACING RATIONALE:
%   - 1.35× (135%) exceeds minimum 1.5× recommended by some guidelines
%   - but 1.5× was found excessive for academic text
%   - 1.35× balances readability with page economy
%
% AVOIDING PROBLEMATIC ELEMENTS:
%   - No justified text with hyphenation (causes "rivers")
%   - No pure black text on pure white (too high contrast)
%   - Minimal italics (hard for dyslexic readers)
%   - No all-caps headings (reduces recognition)
%
% ============================================================================


% ============================================================================
% ADDITIONAL TYPOGRAPHY SETTINGS
% ============================================================================
% Allow flexible line breaking to prevent overfull hbox warnings
\tolerance=9999
\emergencystretch=3em
\hfuzz=\maxdimen % Suppress all cosmetic overfull hbox warnings
\vfuzz=\maxdimen % Suppress all cosmetic vbox warnings
\hbadness=10000 % Suppress underfull hbox warnings
\vbadness=10000 % Suppress underfull vbox warnings

% ============================================================================
% DOCUMENT METADATA
% ============================================================================
\title{QGP Signatures in Light-Ion Collisions (O--O and Ne--Ne) at the LHC:\\[0.5\baselineskip] A Theoretical and Experimental Synthesis}
\author{Deirikr Jaiusadastra Afrauthihinngreygaard\\[0.3\baselineskip]\normalsize\textit{Independent Researcher}}
\date{\today\\[0.5\baselineskip]\footnotesize Document version: \today\ build}

% --- DOCUMENT ---
\begin{document}

\maketitle

% --- ABSTRACT ---
\begin{abstract}
\noindent
The first oxygen--oxygen (O--O) and neon--neon (Ne--Ne) collisions at $\sqrt{s_{NN}} = 5.36$~TeV (LHC, July 2025) probe quark--gluon plasma (QGP) formation in systems intermediate between proton--nucleus and heavy-nucleus collisions. This synthesis reviews the theoretical framework and initial experimental results. \textbf{Key measurements}: CMS reports inclusive O--O $R_{AA}^{\min} \approx 0.69 \pm 0.04$ at $p_T \approx 6$~GeV (suppressed \emph{to} $\sim$70\% of the binary-scaled pp reference, i.e., $\sim$31\% suppression); Ne--Ne shows stronger suppression with $R_{AA}^{\min} \approx 0.65$. ALICE observes geometry-driven anisotropic flow with Ne--Ne/O--O $v_2$ ratio $\approx 1.08$ in ultracentral collisions, consistent with $^{20}$Ne's prolate deformation ($\beta_2 \approx 0.45$). These results provide strong evidence that jet quenching and collective flow---hallmarks of QGP---emerge in systems with $dN_{ch}/d\eta \gtrsim 100$. \textbf{Scope}: Throughout this document, claims are classified as \textsc{[Data]} (direct measurements), \textsc{[Model]} (theory-constrained predictions), or \textsc{[Schematic]} (illustrative/pedagogical). Transport properties such as $\eta/s$ are model-consistent inferences, not direct extractions. Future measurements (strangeness, femtoscopy, direct photons) will further constrain the QGP onset threshold and its transport coefficients.
\end{abstract}

\vspace{1\baselineskip}

% --- TABLE OF CONTENTS ---
\tableofcontents
\clearpage

% --- LIST OF FIGURES ---
\listoffigures
\clearpage

% The main body is generated from Markdown and placed in build/
% Suppress minor overfull warnings from Pandoc-generated tables
\begingroup
\hfuzz=\maxdimen
\input{build/body.tex}
\endgroup


% --- FIGURES ---
% Comprehensive physics-based visualizations compiled from figures/*.tex

% Figure 1: QCD Phase Diagram (December 2025 Update)
\begin{figure}[p]
    \centering
    \includegraphics[width=0.98\textwidth, height=0.88\textheight, keepaspectratio]{build/figures/qcd_phase_diagram.pdf}
    \caption[QCD phase diagram with December 2025 lattice constraints]{%
    \textbf{[Model + Data]} The QCD phase diagram. Gray: CP excluded at $\mu_B < 450$~MeV (2$\sigma$, Bors\'{a}nyi et al.\ PRD 112, 2025). FRG consensus CEP at $(T{=}110, \mu_B{=}630)$~MeV shown with uncertainty. First-order consensus band encompasses NJL/PQM/FRG theoretical spread. Crossover: HotQCD (2019), $\kappa_2 = 0.012(2)$. Freeze-out: Lysenko et al.\ PRC 111 (2025).}
    \label{fig:phase_diagram}
\end{figure}

% Figure 2: Nuclear Structure Comparison
\begin{figure}[htbp]
    \centering
    \includegraphics[width=0.95\textwidth]{build/figures/nuclear_structure.pdf}
    \caption[Nuclear structure: Woods-Saxon profiles, $^{16}$O clustering, $^{20}$Ne deformation]{\textbf{[Schematic + Model Input]} Nuclear structure comparison. (a) Woods-Saxon radial density profiles $\rho(r)/\rho_0$ for $^{16}$O, $^{20}$Ne, and $^{208}$Pb, showing the dramatic size difference. (b) Nuclear radius scaling with mass number $A$, following $R_0 = r_0 A^{1/3}$ with $r_0 \approx 1.25$ fm. (c) $^{16}$O schematic showing possible $\alpha$-cluster tetrahedral structure---this configuration is theoretically motivated but its manifestation in high-energy collisions remains under investigation. (d) Prolate-deformed $^{20}$Ne with quadrupole deformation parameter $\beta_2 \approx 0.45$ (from electron scattering and nuclear spectroscopy); this deformation produces the enhanced $v_2$ observed in central Ne--Ne collisions relative to O--O. \textit{Note}: Subnucleonic fluctuations, not shown, may dominate over nuclear-scale deformation at LHC energies.}
    \label{fig:nuclear_structure}
\end{figure}

% Figure 3: Multi-system R_AA comparison
\begin{figure}[htbp]
    \centering
    \includegraphics[width=0.9\textwidth]{build/figures/RAA_multisystem.pdf}
    \caption[$R_{AA}$ vs $p_T$: O--O, Ne--Ne (CMS min-bias), Pb--Pb data and models]{\textbf{[Data + Model]} Nuclear modification factor $R_{AA}$ versus transverse momentum $p_T$ for collision systems ranging from pp (no modification) to central Pb--Pb (strong suppression). The O--O data shown are from \emph{minimum-bias} collisions (CMS, 6.1 nb$^{-1}$), exhibiting a local minimum of $R_{AA}^{\min} \approx 0.69 \pm 0.04$ at $p_T \approx 6$ GeV. Ne--Ne (minimum-bias, 0.76 nb$^{-1}$) shows stronger suppression with $R_{AA}^{\min} \approx 0.65$ at comparable $p_T$. The Pb--Pb curve (0--10\% central, ALICE) serves as a reference for large-system quenching. Theoretical curves assume BDMPS-Z energy loss scaling $\Delta E \propto \hat{q}L^2$.}
    \label{fig:raa_multi}
\end{figure}

% Figure 4: Energy Loss Path Length Dependence
\begin{figure}[htbp]
    \centering
    \includegraphics[width=0.85\textwidth]{build/figures/energy_loss_path.pdf}
    \caption[Parton energy loss vs path length $L$: radiative and collisional components]{\textbf{[Schematic/Theoretical]} Parton energy loss $\langle\Delta E\rangle$ as a function of path length $L$ through the QGP medium, decomposed into radiative (BDMPS-Z formalism, $\propto \hat{q}L^2$) and collisional ($\propto L$) components. Vertical dashed lines indicate \emph{estimated} typical path lengths for O--O ($\sim$2.5 fm) and Pb--Pb ($\sim$6 fm) systems; actual path-length distributions depend on centrality and geometry. The transport coefficient $\hat{q} \approx 1.5$--$4.5$ GeV$^2$/fm is constrained by JET Collaboration global fits to Pb--Pb data. The curves illustrate the qualitative scaling that explains why O--O exhibits measurable but moderate jet quenching compared to Pb--Pb; quantitative extraction of $\langle\Delta E\rangle$ in O--O awaits detailed model comparisons.}
    \label{fig:energy_loss}
\end{figure}

% Figure 5: Comprehensive Flow Analysis
\begin{figure}[htbp]
    \centering
    \includegraphics[width=0.95\textwidth]{build/figures/flow_comprehensive.pdf}
    \caption[Anisotropic flow $v_2$, $v_3$ vs centrality: O--O, Ne--Ne, Pb--Pb]{\textbf{[Data + Schematic]} Comprehensive analysis of anisotropic flow. (a) Elliptic flow $v_2$ versus centrality for O--O, Ne--Ne, and Pb--Pb: the enhanced $v_2$ in central Ne--Ne relative to O--O is consistent with ALICE observations of geometry-driven flow from Ne's prolate deformation. (b) Triangular flow $v_3$ versus centrality: in small systems, $v_3$ increases toward central collisions due to dominant initial-state fluctuations (CMS, ALICE). (c) \emph{Schematic} azimuthal particle distribution $dN/d\phi$ illustrating the characteristic modulation; numerical values shown are illustrative. (d) \emph{Schematic} of the geometry-to-flow transformation concept.}
    \label{fig:flow_comprehensive}
\end{figure}

% Figure 6: Two-Particle Correlation Ridge
\begin{figure}[htbp]
    \centering
    \includegraphics[width=0.9\textwidth]{build/figures/correlation_2d_ridge.pdf}
    \caption[Two-particle correlations $C(\Delta\eta, \Delta\phi)$ showing ridge structure]{\textbf{[Schematic]} Two-particle correlation function $C(\Delta\eta, \Delta\phi)$ in central Pb--Pb collisions, revealing the characteristic ``ridge'' structure indicative of collective flow. The near-side jet peak at $(\Delta\eta, \Delta\phi) \approx (0, 0)$ arises from jet fragmentation. The long-range ridge extending across $\Delta\eta$ at $\Delta\phi \approx 0$ and $\pi$ is a hallmark of anisotropic collective expansion, arising from $v_2\cos(2\Delta\phi) + v_3\cos(3\Delta\phi)$ modulation. Similar ridge structures observed in high-multiplicity pp, p--Pb, and O--O collisions provided the first evidence for collective behavior in small systems.}
    \label{fig:correlation_ridge}
\end{figure}

% Figure 7: Strangeness Enhancement
\begin{figure}[htbp]
    \centering
    \includegraphics[width=0.9\textwidth]{build/figures/strangeness_enhancement.pdf}
    \caption[Strangeness enhancement vs multiplicity (ALICE data; O--O expectation)]{\textbf{[pp/p--Pb/Pb--Pb: ALICE Data; O--O: Expectation]} Strangeness enhancement versus charged-particle multiplicity $\langle dN_{\mathrm{ch}}/d\eta \rangle$, showing the canonical-to-grand-canonical transition. Single-strange hadrons (K, $\Lambda$) show modest enhancement ($\sim$2$\times$), while multi-strange baryons ($\Xi$, $\Omega$) show dramatic enhancement ($\sim$5--10$\times$) as system size increases (ALICE, Nature Phys.\ 13 (2017) 535). The shaded bands indicate multiplicity ranges for different collision systems. \textbf{O--O strangeness data are not yet published}; the O--O band represents the expected multiplicity range, with strangeness yields predicted to follow the established scaling.}
    \label{fig:strangeness}
\end{figure}

% Figure 8: Transverse Momentum Spectra
\begin{figure}[htbp]
    \centering
    \includegraphics[width=0.9\textwidth]{build/figures/spectra_1d_pt.pdf}
    \caption[Transverse momentum spectra for $\pi$, K, p (Blast-Wave model)]{\textbf{[Model]} Transverse momentum spectra $\frac{1}{2\pi p_T}\frac{d^2N}{dp_T dy}$ for identified hadrons ($\pi$, K, p) from the Blast-Wave model, illustrating characteristic features of collective expansion. At low $p_T$, the mass ordering ($\pi > K > p$) reflects phase-space suppression for heavier particles. At intermediate $p_T$, radial flow boosts heavier particles to higher momenta, flattening their spectra relative to pions. The thermal reference curve (gray dashed) shows a pure exponential $e^{-p_T/T}$ with $T = 300$ MeV. The deviation from thermal behavior at low $p_T$ and the convergence of all species at high $p_T$ (where fragmentation dominates) are hallmarks of collective radial flow. Typical Blast-Wave parameters for central Pb--Pb: kinetic freeze-out $T_{\mathrm{kin}} \approx 100$ MeV, surface velocity $\beta_s \approx 0.65$.}
    \label{fig:spectra_pt}
\end{figure}

% Figure 9: Boost-Invariant Spacetime Evolution
\begin{figure}[htbp]
    \centering
    \includegraphics[width=0.9\textwidth]{build/figures/bjorken_spacetime.pdf}
    \caption[Boost-invariant spacetime evolution in $(\tau, \eta_s)$ coordinates]{\textbf{[Schematic]} Boost-invariant longitudinal spacetime evolution of a heavy-ion collision, visualized in proper time-rapidity ($\tau$, $\eta_s$) coordinates. Hyperbolic curves show constant proper-time $\tau = \sqrt{t^2 - z^2}$ surfaces. The system evolves through distinct phases: initial collision, thermalization ($\tau_0 \sim 1$ fm/$c$), QGP expansion, hadronization, and freeze-out. Temperature decreases according to boost-invariant cooling: $T(\tau) = T_0(\tau_0/\tau)^{1/3}$, from $T_0 \sim 300$ MeV to $T_f \sim 100$ MeV. Light cones (dashed) indicate causal structure.}
    \label{fig:bjorken}
\end{figure}

% Figure 10: Temperature Evolution
\begin{figure}[htbp]
    \centering
    \includegraphics[width=0.9\textwidth]{build/figures/temperature_1d_evolution.pdf}
    \caption[Temperature evolution $T(\tau)$ for Pb--Pb and O--O: ideal vs viscous hydro]{\textbf{[Model]} Temperature evolution $T(\tau)$ as a function of proper time for Pb--Pb (red) and O--O (blue) collisions, comparing ideal ($T \propto \tau^{-1/3}$) and viscous hydrodynamics. The pseudocritical temperature $T_c = 156.5 \pm 1.5$ MeV (blue band) marks the QGP-to-hadron crossover, while kinetic freeze-out occurs at $T_{\mathrm{kin}} \approx 100$ MeV (green band). Viscous corrections slow the cooling rate due to entropy production from shear stress. The QGP lifetime---the interval between thermalization ($\tau_0 \approx 0.6$ fm/$c$) and hadronization ($T = T_c$)---is $\sim$10 fm/$c$ for Pb--Pb but only $\sim$4 fm/$c$ for O--O, explaining the reduced but measurable jet quenching in smaller systems.}
    \label{fig:temperature_evolution}
\end{figure}

% Figure 11: Glauber Monte Carlo Event Display
\begin{figure}[htbp]
    \centering
    \includegraphics[width=0.85\textwidth]{build/figures/glauber_event_display.pdf}
    \caption[Glauber MC event display: participants and spectators in $(x,y)$ plane]{\textbf{[Schematic]} Glauber Monte Carlo event display for a non-central heavy-ion collision in the transverse ($x$--$y$) plane. Nucleon positions are sampled from Woods-Saxon distributions. \textcolor{red}{Red}: participant nucleons in the overlap region that undergo inelastic collisions. \textcolor{gray}{Gray}: spectator nucleons that continue along the beam axis. The almond-shaped overlap region (yellow) determines the initial geometry for QGP formation. Impact parameter $b$ controls the centrality class.}
    \label{fig:glauber}
\end{figure}

% Figure 8: Initial Energy Density
\begin{figure}[htbp]
    \centering
    \includegraphics[width=0.85\textwidth]{build/figures/energy_density_2d.pdf}
    \caption[Initial energy density $\varepsilon(x,y)$ with hot spots and eccentricity]{\textbf{[Schematic]} Initial energy density $\varepsilon(x,y)$ in the transverse plane for a non-central collision ($b \approx 6$ fm). The characteristic almond shape reflects the overlap geometry of two nuclei. Granular hot spots from individual nucleon--nucleon collisions create initial-state fluctuations that drive triangular flow $v_3$. The spatial eccentricity $\varepsilon_2$ (indicated by arrows) is converted to momentum anisotropy $v_2$ via hydrodynamic evolution. Colorbar shows energy density in GeV/fm$^3$.}
    \label{fig:energy_density}
\end{figure}

% Figure 9: Femtoscopy/HBT Correlations
\begin{figure}[htbp]
    \centering
    \includegraphics[width=0.95\textwidth]{build/figures/femtoscopy_hbt.pdf}
    \caption[Femtoscopy/HBT: $C(q)$ and radii vs multiplicity (ALICE; O--O extrapolated)]{\textbf{[pp/Pb--Pb: Data; O--O: Extrapolation]} Femtoscopy (HBT) analysis of the QGP fireball. (a) Two-pion correlation function $C(q)$ versus relative momentum $q$: pp and Pb--Pb curves are based on ALICE measurements; the O--O curve is an extrapolation based on multiplicity scaling. (b) HBT radii ($R_{\mathrm{out}}$, $R_{\mathrm{side}}$, $R_{\mathrm{long}}$) versus charged-particle multiplicity, demonstrating the universal $R \propto (dN_{\mathrm{ch}}/d\eta)^{1/3}$ scaling established by ALICE across collision systems. (c) $R_{\mathrm{out}}/R_{\mathrm{side}}$ ratio versus centrality. (d) Spacetime homogeneity regions showing the characteristic increase in fireball volume. \textbf{O--O femtoscopy measurements at 5.36 TeV are anticipated but not yet published}; the O--O points shown are extrapolated expectations.}
    \label{fig:femtoscopy}
\end{figure}

% Figure 10: Direct Photon Spectra
\begin{figure}[htbp]
    \centering
    \includegraphics[width=0.9\textwidth]{build/figures/direct_photon_spectra.pdf}
    \caption[Direct photon spectra: prompt + thermal (Pb--Pb data; O--O model)]{\textbf{[Pb--Pb: Data; O--O: Model Prediction]} Direct photon spectra and thermal radiation from the QGP. The total direct photon yield (solid lines) is decomposed into prompt photons (dashed, power-law $\propto p_T^{-n}$) from initial hard scatterings and thermal photons (dotted, exponential $\propto e^{-p_T/T_{\mathrm{eff}}}$) from the QGP medium. \textbf{Pb--Pb effective temperatures} ($T_{\mathrm{eff}} \approx 220$--$280$ MeV, ALICE/PHENIX) are experimentally established. \textbf{O--O photon thermometry has not yet been measured}; the O--O curve shown is a theoretical expectation scaled from Pb--Pb by system-size arguments. Direct photon measurements in O--O represent a future experimental goal requiring high integrated luminosity due to the small signal-to-background ratio in lighter systems. The ``direct photon puzzle''---observed $v_2^{\gamma}$ exceeding theoretical predictions---remains an active area of investigation.}
    \label{fig:photons}
\end{figure}

% Figure 15: Knudsen Number and QGP Threshold
\begin{figure}[htbp]
    \centering
    \includegraphics[width=0.95\textwidth]{build/figures/knudsen_scaling.pdf}
    \caption[Knudsen number vs $R$ and QGP signature onset vs multiplicity]{\textbf{[Schematic/Conceptual]} System-size dependence of QGP signatures and the validity of hydrodynamics. \textbf{Left}: Schematic of Knudsen number $\mathrm{Kn} = \lambda_{\mathrm{mfp}}/R$ versus system radius $R$, where $\mathrm{Kn} \propto (\eta/s)/(TR)$ for a relativistic fluid. The mean free path $\lambda_{\mathrm{mfp}}$ in QGP is model-dependent ($\sim$0.1--0.3 fm); the curve illustrates the qualitative dependence rather than a quantitative prediction. Hydrodynamics is expected to be applicable when $\mathrm{Kn} \lesssim 0.1$. \textbf{Right}: Conceptual ``turn-on'' of QGP signatures versus charged-particle multiplicity $dN_{\mathrm{ch}}/d\eta$. This panel synthesizes the qualitative finding that all signatures (flow, strangeness, suppression, HBT) emerge gradually with increasing multiplicity. The $dN_{\mathrm{ch}}/d\eta \approx 135$ value corresponds to 0--5\% central O--O (CMS), \emph{not} the minimum-bias event class. A rigorous falsifiable criterion for ``hydro validity'' would require specification of acceptable viscous corrections to specific observables.}
    \label{fig:knudsen_threshold}
\end{figure}

% --- APPENDIX: TERMINOLOGY AND NOMENCLATURE ---
\appendix
\section*{Appendix: Terminology and Nomenclature}
\addcontentsline{toc}{section}{Appendix: Terminology and Nomenclature}

The following terms and coordinate systems are used throughout this document. We adopt physics-descriptive terminology where possible, with established eponymous names noted parenthetically.

\subsection*{Coordinate Systems and Kinematics}

\begin{description}
\item[Proper time ($\tau$)] The Lorentz-invariant time coordinate $\tau = \sqrt{t^2 - z^2}$, representing time measured in the local rest frame of a fluid element. Surfaces of constant $\tau$ are hyperbolas in the $(t,z)$ plane.

\item[Spacetime rapidity ($\eta_s$)] The boost-invariant spatial coordinate $\eta_s = \frac{1}{2}\ln\frac{t+z}{t-z} = \mathrm{artanh}(z/t)$. \textit{Not to be confused with} momentum-space rapidity $y$.

\item[Boost-invariant coordinates ($\tau$, $\eta_s$)] Also known as Milne coordinates or (in heavy-ion physics context) Bjorken coordinates. The natural coordinate system for longitudinally boost-invariant expansion, where the metric becomes $ds^2 = d\tau^2 - dx^2 - dy^2 - \tau^2 d\eta_s^2$.

\item[Momentum rapidity ($y$)] For a particle with energy $E$ and longitudinal momentum $p_z$: $y = \frac{1}{2}\ln\frac{E+p_z}{E-p_z}$. Transforms additively under longitudinal boosts.

\item[Pseudorapidity ($\eta$)] The experimentally convenient approximation $\eta = -\ln[\tan(\theta/2)]$ where $\theta$ is the polar angle. Equals rapidity $y$ for massless particles.

\item[Parton momentum fraction ($x$)] Also called Bjorken-$x$, the fraction of the hadron's longitudinal momentum carried by a parton: $x = Q^2/(2P \cdot q)$. Standard notation in deep inelastic scattering.
\end{description}

\subsection*{Energy Density and Thermodynamics}

\begin{description}
\item[Initial energy density ($\varepsilon_{\mathrm{Bj}}$)] The energy density at formation time $\tau_0$, estimated via the boost-invariant formula:
\begin{equation*}
\varepsilon_{\mathrm{Bj}} = \frac{1}{\tau_0 A_\perp} \frac{dE_T}{dy}\bigg|_{y=0}
\end{equation*}
Standard notation $\varepsilon_{\mathrm{Bj}}$ acknowledges Bjorken's original derivation.

\item[Boost-invariant cooling] Temperature evolution under longitudinal expansion: $T(\tau) = T_0(\tau_0/\tau)^{1/3}$ for an ideal relativistic gas with equation of state $\varepsilon = 3P$.

\item[Pseudocritical temperature ($T_c$)] The crossover temperature between hadronic matter and QGP, $T_c = 156.5 \pm 1.5$ MeV from lattice QCD at zero chemical potential.

\item[Chemical freeze-out temperature ($T_{\mathrm{ch}}$)] The temperature at which inelastic hadron scattering ceases, $T_{\mathrm{ch}} \approx 156$ MeV.

\item[Kinetic freeze-out temperature ($T_{\mathrm{fo}}$)] The temperature at which elastic scattering ceases, $T_{\mathrm{fo}} \approx 100$--$120$ MeV.
\end{description}

\subsection*{Transport Coefficients}

\begin{description}
\item[Specific shear viscosity ($\eta/s$)] Shear viscosity normalized by entropy density. The QGP exhibits $\eta/s \approx 0.08$--$0.16$, near the conjectured lower bound $\eta/s \geq 1/(4\pi) \approx 0.08$ from holographic gauge/gravity duality (AdS/CFT correspondence).

\item[Jet transport coefficient ($\hat{q}$)] The mean transverse momentum squared transferred to a hard parton per unit path length: $\hat{q} = \langle k_\perp^2 \rangle/L$. Characterizes the ``stopping power'' of the QGP for jets.
\end{description}

\subsection*{Flow and Anisotropy}

\begin{description}
\item[Azimuthal anisotropy ($v_n$)] Fourier coefficients of the particle azimuthal distribution: $\frac{dN}{d\phi} \propto 1 + 2\sum_n v_n \cos[n(\phi - \Psi_n)]$, where $\Psi_n$ is the event plane angle.

\item[Elliptic flow ($v_2$)] Quantifies the second-harmonic azimuthal modulation, driven by the initial spatial eccentricity $\varepsilon_2$.

\item[Triangular flow ($v_3$)] Third-harmonic modulation driven by initial-state fluctuations rather than collision geometry.
\end{description}

\subsection*{Femtoscopy and HBT}

\begin{description}
\item[HBT radii ($R_{\mathrm{out}}$, $R_{\mathrm{side}}$, $R_{\mathrm{long}}$)] Hanbury Brown--Twiss correlation radii extracted from two-particle correlations, characterizing the spacetime extent of the homogeneity region.

\item[Homogeneity region] The spacetime volume from which identical particles are emitted with similar momenta; HBT radii probe this region's dimensions.
\end{description}

% --- BIBLIOGRAPHY ---
% Include all references (content uses inline citations, not \cite{})
\nocite{*}
\bibliographystyle{unsrtnat}
\bibliography{references}

\end{document}
